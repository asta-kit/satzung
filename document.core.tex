% Muster für die Satzungsdokumente

%\stupadate{} % Hier trägt man das Datum des Beschlusses im StuPa ein
%\publishdate{} % Hier trägt man das Datum der Bekanntmachung/des Inkrafttreten ein
%\jurchanges{
	% Hier trägt man die Änderungen ein die eingearbeitet wurden
	%\item Amtliche Bekanntmachung 2013 KIT 004 vom 04.02.2013
	%\item Amtliche Bekanntmachung 2014 KIT 021 vom 02.05.2014
%}

% Hier kommt der Titel des Dokuments. In eckigen Klammern falls benötigt ein
% Kurztitel der dann statt des langen in Kopfzeilen oder im Inhaltsverzeichnis
% angezeigt wird
\jurtitle[Kurztitel]{Langer Titel\\des Dokuments}

% Hier beginnt das eigentliche Dokument
\begin{jurdoc}

\jurparagraph{Erster Paragraph}\label{neues_dokument:erster_paragraph}
% Die Labels sind nützlich für Verweise, da kann man dann mit \ref{} drauf verweisen
Erster Absatz

Zweiter Absatz\label{neues_dokument:erster_paragraph:toller_absatz}


\jurparagraph{Noch ein Paragraph}\label{neues_dokument:anderer_paragraph}
Und hier kommt jetzt ein Verweis auf den ersten Paragraphen \ref{neues_dokument:erster_paragraph} und noch spezieller kann man direkt auf den Absatz verweisen \ref{neues_dokument:erster_paragraph:toller_absatz}.

Dieser Absatz hat jetzt eine Aufzählung:
\begin{enumerate}
	\item Erstes Ding,
	\item Zweites Ding und
	\item das letzte Ding.
\end{enumerate}



% Hier ist das Dokument zu Ende
\end{jurdoc}
