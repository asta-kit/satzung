\jurtitle[Wahl- und Abstimmungsordnung]{Wahl- und Abstimmungsordnung \\ der Verfassten Studierendenschaft \\ des Karlsruher Instituts für Technologie (KIT)}

\begin{jurdoc}

\Paragraph{title={Geltungsbereich}}\label{wahl:geltungsbereich}
Diese Satzung regelt die Wahlen zum Studierendenparlament und den Fachschaftsvorständen, die Urabstimmung, sowie die Wahlen zu weiteren Organen der Studierendenschaft, sofern eine Satzung dies vorsieht.

\Paragraph{title={Wahlsystem}}\label{wahl:wahlsystem}
Das Studierendenparlament wird nach Listen, welche aufgrund gültiger Wahlvorschläge aufgestellt werden, gewählt. 

Bei der Wahl des Studierendenparlament hat jede Wahlberechtigte
\begin{enumerate}
    \item eine Stimme, mit welcher sie eine Liste wählen kann (Listenstimme),
    \item fünf Stimmen mit denen sie Kandidatinnen wählen kann (Personenstimmen); die Stimmen können beliebig kumuliert und panaschiert werden.
\end{enumerate}

Die Fachschaftsvorstände werden in Persönlichkeitswahl von den Fachschaftsmitgliedern gewählt. Die Benennung der Kandidatinnen erfolgt durch die entsprechende Fachschaftsversammlung.

Bei der Wahl der Fachschaftsvorstände hat jede Wählerin so viele Stimmen wie Fachschaftssprecherinnen zu wählen sind. Auf eine Kandidatin dürfen dabei maximal zwei Stimmen kumuliert werden.

\Paragraph{title={Urabstimmung}}\label{wahl:urabstimmung}
Für die Urabstimmung gelten die Regelungen zur Durchführung der Wahlen sinngemäß. 

Die Urabstimmung findet spätestens während der nächsten Wahl zum Studierendenparlament statt. Der Antrag auf Urabstimmung kann den Zeitraum für die Urabstimmung beinhalten. Dabei muss die Einhaltung der Fristen gewährleistet sein.

\Paragraph{title={Wahlberechtigung und Wählbarkeit}}\label{wahl:wahlberechtigung}

Wahlberechtigt sind alle Mitglieder, die im Wählerverzeichnis eingetragen sind.

Wählbar sind alle Mitglieder, welche in einem gültigen Wahlvorschlag aufgeführt sind. Mitglieder des Wahlausschusses sowie des Ältestenrates dürfen in keinen Wahlvorschag aufgenommen werden.

\Paragraph{title={Wahltermin}}\label{wahl:walhtermin}
Die Wahlen sollen jährlich im Sommersemester parallel zur Wahl der studentischen Senatsmitglieder des KIT stattfinden. 

Das Studierendenparlament legt Termin und Dauer der Wahlen fest. Der Wahlzeitraum besteht aus wenigstens drei und höchstens fünf aufeinanderfolgenden Tagen. Die Wahlen dürfen nicht in den ersten drei Vorlesungswochen oder in der letzten Vorlesungswoche des Semesters stattfinden. 

In der Regel finden die Wahlen zum Studierendenparlament und den Fachschaftsvorständen zeitgleich statt.

\Paragraph{title={Wahlausschuss}}\label{wahl:wahlausschuss}
Das Studierenenparlament wählt spätestens 48 Tage vor dem ersten Wahltag bzw. spätestens 27 Tage vor dem ersten Tag der Urabstimmung einen aus vier Personen bestehenden Wahlausschuss. 

Der Wahlausschuss wählt eine Vorsitzende sowie eine Stellvertreterin aus seiner Mitte.

Der Wahlausschuss ist beschlussfähig, wenn mehr als die Hälfte seiner Mitglieder anwesend sind. Beschlüsse werden mit der Mehrheit der anwesenden Mitglieder getroffen. Bei Stimmgleichheit entscheidet die Stimme der Vorsitzenden. Über die Sitzungen ist Protokoll zu führen.

Der Wahlausschuss ist zuständig für
\begin{enumerate}
    \item Bekanntmachung der Wahl bzw. Urabstimmung,
    \item Erstellung des Wählerverzeichnisses,
    \item Annahme, Zulassung und Bekanntmachung der Wahlvorschläge,
    \item Anfertigung der Stimmzettel sowie der weiteren für Wahl und Auszählung erforderlichen Unterlagen,
    \item Beschaffung, Versiegelung und Aufbewahrung der Wahlurnen,
    \item Organisation und Durchführung der Wahl bzw. Urabstimmung sowie der Auszählung,
    \item Feststellung und Bekanntmachung der Wahlergebnisse,
    \item die Einhaltung demokratischer Regeln.
\end{enumerate}

\Paragraph{title={Bekanntmachung der Wahlen}}\label{wahl:bekanntmachung-wahl}
Der Wahlausschuss macht die Wahl spätestens 42 Tage vor dem ersten Wahltag bekannt.

Die Bekanntmachung enthält:
\begin{enumerate}
     \item den Wahlzeitraum sowie die Abstimmungszeiten,
     \item den Hinweis, dass nur wählen darf, wer am Tag des endgültigen Abschlusses des Wählerverzeichnises in diesem eingetragen ist,
     \item Ort, Dauer und Zeit der Einsichtsmöglichkeit in das Wählerverzeichnisses,
     \item die zu wählenden Gremien sowie die Zahl der jeweils zu wählenden Mitglieder und deren Amtszeit,
     \item die Aufforderung, spätestens am 24. Tag vor dem ersten Wahltag Wahlvorschläge beim Wahlausschusses einzureichen, 
     \item den Hinweis, dass Wahlbewerber und Vertreter eines Wahlvorschlags nicht Mitglieder des Wahlausschusses sein können,
     \item den Hinweis auf die Form und den Inhalt der Wahlvorschläge,
     \item die Bestimmungen über die Briefwahl nach \ref{wahl:briefwahl},
     \item den Hinweis darauf, dass keine Bindung an eine bestimmte Wahlurne besteht,
     \item den Hinweis darauf, wo die Wahl- und Abstimmungsordnung einzusehen ist.
\end{enumerate}

Die Bekanntmachung ist nicht vor Ende der Wahl zu entfernen.

\Paragraph{title={Bekanntmachung der Urabstimmung}}\label{wahl:bekanntmachung-ua}
Der Wahlausschuss macht die Urabstimmung spätestens 21 Tage vor dem ersten Wahltag bekannt.

Die Bekanntmachung enthält:
\begin{enumerate}
     \item den Antragstext sowie die Abstimmungsmöglichkeiten,
     \item den Wahlzeitraum sowie die Abstimmungszeiten,
     \item den Hinweis, dass nur wählen darf, wer am Tag des endgültigen Abschlusses des Wählerverzeichnises in diesem eingetragen ist,
     \item Ort, Dauer und Zeit der Einsichtsmöglichkeit in das Wählerverzeichnisses,
     \item die Bestimmungen über die Briefwahl nach \ref{wahl:briefwahl},
     \item den Hinweis darauf, dass keine Bindung an eine bestimmte Wahlurne besteht,
     \item den Hinweis darauf, wo die Wahl- und Abstimmungsordnung einzusehen ist.
\end{enumerate}

Die Bekanntmachung ist nicht vor Ende der Urabstimmung zu entfernen.

\Paragraph{title={Wählerverzeichnis}}\label{wahl:wählernverzeichnis}

Alle Wahlberechtigten sind in ein Wählerverzeichnis in Listenform einzutragen. Die Aufstellung des Wählerverzeichnisses obliegt dem  Wahlausschuss. Es kann im Wahlverfahren auch in elektronischer Form  verwendet werden.

Das Wählerverzeichnis enthält die folgenden Angaben:
\begin{enumerate}
\item laufende Nummer,
\item Familienname,
\item Vorname,
\item Matrikelnummer,
\item Studiengang,
\item Vermerk über die Stimmabgabe,
\item Bemerkungen.
\end{enumerate}

Das Wählerverzeichnis ist sieben Tage nach Bekanntmachung der Wahl bzw. Urabstimmung vorläufig  abzuschließen und für fünf Tage beim Wahlausschuss zur Einsicht durch die Studierenden aufzulegen. Eine Einsichtnahme steht jedem zu, um seine  eigenen Daten auf Richtigkeit und Vollständigkeit zu überprüfen. Zur Überprüfung der Richtigkeit oder Vollständigkeit der Daten von anderen im Wählerverzeichnis eingetragenen Personen haben Wahlberechtigte nur dann ein Recht auf Einsicht in das Wählerverzeichnis, wenn sie Tatsachen glaubhaft machen, aus denen sich eine Unrichtigkeit oder Unvollständigkeit des Wählerverzeichnisses ergeben kann.  \label{wahl:wählerverzeichnis:auflegung}

Das Wählerverzeichnis ist spätestens am 14. Tag nach Bekanntmachung der Wahl bzw. Urabstimmung  unter Berücksichtigung der Entscheidungen nach  \ref{wahl:aenderungwv:aenderungen} vom Wahlausschuss endgültig  abzuschließen. Dabei ist im Wählerverzeichnis
\begin{enumerate}
\item die Zahl der eingetragenen Wahlberechtigten,
\item die Zahl der Anträge auf Berichtigung des Wählerverzeichnisses
\end{enumerate}
vom Wahlausschuss zu beurkunden.

\Paragraph{title={Änderung des Wählerverzeichnisses}}
Das Wählerverzeichnis kann bis zum Ablauf der Einsichtsfrist vom Wahlausschuss berichtigt oder ergänzt werden.

Die  Einsichtsberechtigten gemäß \ref{wahl:wählerverzeichnis:auflegung}  können während der Dauer der Auflegung des Wählerverzeichnisses dessen  Berichtigung oder Ergänzung beantragen, wenn sie diese für unrichtig  oder unvollständig halten. Der Antrag ist schriftlich beim Wahlausschuss  zu stellen. Die erforderlichen Beweise sind vom Antragsteller  beizubringen. Der Wahlausschuss entscheidet spätestens am 14. Tag nach Bekanntmachung der Wahl bzw. Urabstimmung  über die Anträge. Die Entscheidung ist der Antragstellerin und ggf. der  Betroffenen mitzuteilen. \label{wahl:aenderungwv:aenderungen}

Nach  Ablauf der Einsichtsfrist bis zum endgültigen Abschluss des  Wählerverzeichnisses können Eintragungen und Streichungen nur in Vollzug  der Entscheidung gemäß \ref{wahl:aenderungwv:aenderungen} vorgenommen  werden.

Das  Wählerverzeichnis kann bis zum 1. Tag vor dem ersten Wahltag vom  Wahlausschuss bei Vorliegen offensichtlicher Fehler, Unstimmigkeiten  oder Schreibversehen berichtigt oder ergänzt werden.
%soll das "bis zum Tag vor dem ersten Wahltag" heißen?

Änderungen sind als solche kenntlich zu machen und mit Datum und Unterschrift zu versehen.

\Paragraph{title={Wahlvorschläge}}\label{wahl:wahlvorschlaege}

Die Wahlvorschläge sind getrennt für die Wahlen zum Studierendenparlament und den Fachschaftsvorständen spätestens am 24. Tag vor dem ersten Wahltag bis 15:00 Uhr beim Wahlausschuss einzureichen.

Wahlvorschläge für die Wahl zum Studierendenparlament müssen enthalten
\begin{enumerate}
    \item ein Kennwort; Kennwörter dürfen nicht irreführend sein,
    \item eine Liste mit Kandidatinnen; ein Wahlvorschlag darf höchstens so viele Kandidatinnen enthalten, wie Plätze im Studierendenparlament zu besetzen sind, 
    \item eine von mindestens 30 Wahlberechtigten unterzeichnete Unterstützungsliste.
\end{enumerate}

Geben die Kennwörter mehrerer Wahlvorschläge zu Verwechslungen Anlass, so fordert der Wahlausschuss die Vertreterin des später eingereichten Wahlvorschlages unverzüglich auf, sich innerhalb der Mängelbeseitigungsfrist ein anderes Kennwort zu geben.

Ein Wahlvorschlag für den Fachschaftsvorstand wird von der Fachschaftsversammlung erstellt. Er wird von der amtierenden Fachschaftsleiterin unterzeichnet und vertreten. Er beinhaltet:
\begin{enumerate}
    \item eine Liste mit Kandidatinnen,
    \item eine von Sitzungsleitung und Protokollantin unterzeichnete Kopie des Protokolls der Fachschaftsversammlung.
\end{enumerate}

Unterzeichner müssen für die betreffende Wahl wahlberechtigt sein. Sie müssen folgende Angaben machen:
\begin{enumerate}
\item Vor- und Familienname,
\item Matrikelnummer,
\item eigenhändige Unterschrift,
\item bei den ersten beiden Unterstützerinnen: E-Mailadresse und Telefonnummer.
\end{enumerate}
Die erste Unterzeichnerin ist zur Vertretung gegenüber dem Wahlausschuss berechtigt, die zweite Unterzeichnerin vertritt sie.

Eine Wahlberechtigte darf für dieselbe Wahl nicht mehr als einen Wahlvorschlag unterzeichnen. Hat eine Wahlberechtigte dies nicht beachtet, so wird sie von allen eingereichten Wahlvorschlägen gestrichen.

Die Liste der Kandidatinnen muss folgende Angaben zu den Kandidatinnen enthalten:
\begin{enumerate}
\item Laufende Nummer,
\item Vor- und Familienname,
\item Matrikelnummer,
\item Studiengang,
\item E-Mailadresse,
\item eigenhändige Unterschrift.
\end{enumerate}
Die Kandidatinnen bestätigen mit ihrer Unterschrift die Richtigkeit der Daten sowie ihre Zustimmung, auf den Wahlvorschlag aufgenommen zu werden. Eine Kandidatin darf nicht auf mehreren Wahlvorschlägen für dieselbe Wahl aufgenommen werden.

Mitglieder des Wahlausschusses sowie des Ältestenrats dürfen weder auf einem Wahlvorschlag als Kandidatin geführt werden noch einen vertreten.

Die Zurücknahme von Wahlvorschlägen, Unterschriften unter einem Wahlvorschlag oder Zustimmungserklärungen von Bewerbern ist nur bis zum Ablauf der Einreichungsfrist für die Wahlvorschläge zulässig.

Etwaige Mängel am Wahlvorschlag sind der Vertreterin des Wahlvorschlages unverzüglich, spätestens aber am Tag nach Ablauf der Einreichungsfrist mitzuteilen. Danach besteht bis zum Beginn der Wahlausschusssitzung nach \ref{wahl:zulassung:sitzung} die Gelegenheit, die Mängel zu beseitigen. Das Fehlen von erforderlichen Unterschriften gilt nicht als Mangel im oberen Sinne. Diese können nach Ablauf der Einreichungsfrist nicht nachgeholt werden.

\Paragraph{title={Zulassung der Wahlvorschläge}}\label{wahl:zulassung}
Spätestens am 21. Tag vor dem ersten Wahltag beschließt der Wahlausschuss in einer Sitzung über die Zulassung der eingereichten Wahlvorschläge. \label{wahl:zulassung:sitzung}

Zurückzuweisen sind Wahlvorschläge, 
\begin{enumerate}
      \item die nicht fristgerecht eingereicht wurden,
      \item die eine Bedingung enthalten,
      \item die nicht von einer ausreichenden Zahl Wahlberechtigter unterzeichnet wurden, 
      \item welche die Reihenfolge oder die Zuordnung der Personendaten der Kandidatinnen nicht zweifelsfrei erkennen lassen.
\end{enumerate}

In den Wahlvorschlägen sind diejenigen Bewerber zu streichen
\begin{enumerate}
\item die so unvollständig bezeichnet werden, dass Zweifel über ihre Person bestehen,
\item die nicht wählbar sind,
\item deren Zustimmungserklärung nicht ordnungsgemäß vorgelegt wurde oder unter einer Bedingung eingegangen ist,
\item deren Zustimmungserklärung vor Ablauf der Einreichungsfrist der Wahlvorschläge zurückgezogen wurde,
\item die in mehreren Wahlvorschlägen für dieselbe Wahl aufgeführt sind.
\end{enumerate}

Überzählige Kandidatinnen werden in der Reihenfolge von hinten gestrichen.

Die Beschlüsse und deren Begründungen sind in ein Protokoll aufzunehmen.\label{wahl:zulassung:beschluss}

Wird ein Wahlvorschlag zurückgewiesen oder eine Kandidatin gestrichen, so sind die getroffenen Entscheidungen der Vertreterin des Wahlvorschlages sowie der betroffenen Kandidatin unverzüglich mitzuteilen.

Der Wahlausschuss bestimmt unter den Wahlvorschlägen zum Studierendenparlament per Losziehung eine Reihenfolge.

\Paragraph{title={Bekanntmachung der Wahlvorschläge }}\label{wahl:bekanntmachungvorschläge}
Der Wahlausschuss macht die Wahlvorschläge spätestens am 14. Tag vor dem ersten Wahltag bekannt. 
Die Bekanntmachung enthält:
\begin{enumerate}
    \item die zu wählenden Gremien sowie die Zahl der jeweils zu wählenden Mitglieder,
    \item die jeweils zugelassenen Wahlvorschläge in der Reihenfolge nach \ref{wahl:zulassung}, 
    \item den Hinweis, dass nur mit den amtlichen Stimmzetteln des Wahlausschusses gewählt werden darf,
    \item den Hinweis auf die den Wahlberechtigten zur Verfügung stehenden Stimmen sowie ggf. den Hinweis auf die Kumulierbarkeit bzw. Panaschierbarkeit der Personenstimmen,
    \item den Wahlzeitraum sowie die Abstimmungszeiten,
    \item den Hinweis darauf, dass keine Bindung an eine bestimmte Urne besteht,
    \item den Hinweis darauf, dass Studentinnen ihre Wahlberechtigung gemäß \ref{wahl:wahlhandlung:nachweis} nachweisen müssen,
    \item den Hinweis darauf, wo die Wahl- und Abstimmungsordnung einzusehen ist.
\end{enumerate}

Die Bekanntmachung ist nicht vor Ende der Wahl zu entfernen.


\Paragraph{title={Briefwahl}}\label{wahl:briefwahl}
Eine Wahlberechtigte, die zum Zeitpunkt der Wahl verhindert ist, die Abstimmung vor Ort vorzunehmen, erhält auf Antrag in Schriftform beim Wahlausschuss für die Wahl einen Wahlschein und die Briefwahlunterlagen, bestehend aus einem Stimmzettel für jede Wahl, einem Wahlumschlag und einem Wahlbriefumschlag. Die Ausgabe der Wahlscheine und der Briefwahlunterlagen ist im Wählerverzeichnis zu vermerken.

Der Wahlumschlag und der Wahlbriefumschlag müssen als solcher gekennzeichnet sein. Weiter muss der Wahlbriefumschlag die Adresse der Wählerin als Absender und die Adresse des Wahlausschusses als Empfänger ausweisen.\\
Die Briefwählerin trägt die Kosten der Übersendung. Sie ist hierauf hinzuweisen.

Briefwahlunterlagen können frühestens am Tag der Bekanntmachung der Wahl und spätestens am 7. Tag (Eingang beim Wahlausschuss) vor dem ersten Wahltag beantragt werden. \label{wahl:briefwahl:frist}

Bei der Briefwahl kennzeichnet die Wählerin ihren Stimmzettel und steckt ihn in den Wahlumschlag. Sie bestätigt auf dem Wahlschein durch Unterschrift, dass sie den beigefügten Stimmzettel persönlich gekennzeichnet hat und legt den Wahlschein mit dem verschlossenen Wahlumschlag in den Wahlbriefumschlag.

Der Wahlbrief ist an die vorgedruckte Anschrift des Wahlausschusses ausreichend frankiert zu übersenden oder persönlich beim Wahlausschuss abzugeben. Der Wahlausschuss kann der Wahlberechtigten die Möglichkeit geben, bei persönlicher Abholung der Briefwahlunterlagen die Briefwahl an Ort und Stelle auszuüben; in diesem Fall kann die Stimmabgabe auch noch nach Ablauf der Frist nach \ref{wahl:briefwahl:frist} erfolgen. Dabei ist Sorge zu tragen, dass der Stimmzettel unbeobachtet gekennzeichnet und in den Wahlumschlag gelegt werden kann. Der Wahlausschuss nimmt sodann den Wahlbrief entgegen.

Die Stimmabgabe gilt als rechtzeitig erfolgt, wenn der Wahlbrief am letzten Wahltag bis spätestens zum Zeitpunkt des Endes der Abstimmungszeit beim Wahlausschuss eingeht. Auf dem Wahlbriefumschlag ist der Tag des Eingangs, auf den am letzten Wahltag eingehenden Wahlbriefumschlägen zusätzlich die Uhrzeit des Eingangs zu vermerken. Sind eingehende Wahlbriefe unverschlossen, so ist dies auf den Wahlbriefen zu vermerken.

Die eingegangenen Wahlbriefe werden vom Wahlausschuss unter Verschluss ungeöffnet aufbewahrt.

Wahlscheine und Wahlumschläge werden gezählt und die Wahlscheine mit den Eintragungen im Wählerverzeichnis verglichen. Wurde von der Wählerin eine Stimmabgabe an der Urne vorgenommen, so ist ihr Wahlumschlag ungeöffnet zu vernichten.

Die Auszählung der per Briefwahl abgegebenen Stimmen erfolgt entsprechend der Auszählung einer Urne gemäß \ref{wahl:auszaehlung}. Haben weniger als zehn Wählerinnen ihre Stimme per Briefwahl abgegeben, so bestimmt der Wahlausschuss eine Urne, zu der die Stimmzettel aus der Briefwahl hinzugefügt werden.

\Paragraph{title={Stimmzettel }}\label{wahl:stimmzettel}

Der Stimmzettel enthält: 
\begin{enumerate}
    \item die zugelassenen Wahlvorschläge mit ihrem Kennwort und den Kandidatinnen mit vollem Namen und Studienfach; bei den Wahlen zu den Fachschaftsvorständen kann auf die Angabe des Studienfachs verzichtet werden.
    \item einen klar zu erkennenden Platz zum Eintragen der Stimmen durch die Wählerinnen,
    \item den Hinweis auf die zur Verfügung stehenden Stimmen sowie den Hinweis auf die Kumulierbarkeit bzw. Panaschierbarkeit der Personenstimmen,
    \item den Hinweis darauf, dass der Stimmzettel vor dem Einwerfen mit dem Aufdruck nach innen zu falten ist,
    \item den Wahlzeitraum.
\end{enumerate}

\Paragraph{title={Wahlurnen und Urnenbuch}}\label{wahl:urnen}
Der Wahlausschuss legt vor Beginn der Wahl die Anzahl der Wahlurnen fest, versiegelt die Urnen und kennzeichnet sie eindeutig und deutlich sichtbar.

Die Urnen sind so einzurichten, dass die eingeworfenen Stimmzettel nicht vor Ende der Wahl entnommen werden können

Die Urnen sind bis zur Auszählung durch Wahlhelferinnen zu beaufsichtigen oder unter Verschluss zu halten.

Zu jeder Urne ist ein Urnenbuch zu führen. Dieses wird vom Wahlausschuss ausgegeben. In das Urnenbuch ist einzutragen: 
\begin{enumerate}
    \item  der volle Name der für die Urne verantwortlichen Wahlhelferin sowie den Zeitraum der Verantwortlichkeit,
    \item die Unterschrift der verantwortlichen Wahlhelferin als Bestätigung, dass sie die Vorschriften der Wahl- und Abstimmungsordnung kennt und danach handelt,
    \item der volle Name aller weiteren Wahlhelferinnen an der Urne,
    \item Zeitpunkt der Öffnung und Schließung der Urne,
    \item der Aufenthaltsort der Urne,
    \item jede während der Wahl festgestellte Unregelmäßigkeit, welche die Urne betrifft, mit dem Zeitpunkt der Feststellung, dem Namen der Feststellenden und der Beschreibung des Vorgangs,
    \item für jede Wählerin den Namen und die Matrikelnummer.
\end{enumerate}

Die Urnen dürfen das Gelände des KIT nicht verlassen; Ausnahmen regelt der Wahlausschuss. Erstreckt sich eine Wahl oder Abstimmung über mehrere Tage, so sind die Urnen über die Nacht von 20:00 bis 7:00 Uhr unter sicherer Verwahrung zu halten. In dieser Zeit ist keine Wahlhandlung zulässig.

\Paragraph{title={ Wahlhelferinnen }}\label{wahl:wahlhelferinnen}
Der Wahlausschuss bestellt Wahlhelferinnen zur Durchführung der Wahl. Der Wahlausschuss belehrt die Wahlhelferinnen über ihre Pflichten. Der Wahlausschuss kann die Bestellung und Belehrung von Wahlhelferinnen an die Wahlleiterinnen gemäß \ref{grundsaetze:wahlen:wahlausschuss} delegieren.

Die Wahlhelferinnen nehmen ihr Amt unparteiisch und gewissenhaft wahr. Sie enthalten sich während der Ausübung ihres Amtes jeder parteilichen Betätigung. Dazu gehört auch das Tragen von Parteiabzeichen und "~parolen.

\Paragraph{title={ Wahlhandlung }}\label{wahl:wahlhandlung}
Jede Urne wird ständig von einer verantwortlichen Wahlhelferin sowie wenigstens einer weiteren Wahlhelferin betreut. Sind unter den Wahlhelferinnen Kandidatinnen, so müssen diese von unterschiedlichen Wahlvorschlägen stammen. Die Wahlhelferinnen sind für die Einhaltung der Wahl- und Abstimmungsordnung an dieser Wahlurne zuständig und dienen als Ansprechpartnerinnen für die Wählerinnen. Sie sind zur gewissenhaften und unparteiischen Auskunft verpflichtet.

Jede Wahlberechtigte kann ihr Stimmrecht für jede Wahl nur einmal und persönlich wahrnehmen. Wahlberechtigte, die durch körperliche Gebrechen gehindert sind, ihre Stimme allein abzugeben, können sich der Hilfe einer Vertrauensperson bedienen.

Die Wählerin weist sich durch Vorlage des Studierendenausweises oder  eines Immatrikulationsnachweises zusammen mit einem amtlichen Lichtbildausweis aus. \label{wahl:wahlhandlung:nachweis}

Die Wahlhelferinnen nehmen die Daten der Wählerin in das Urnenbuch auf und geben dieser die entsprechenden Stimmzettel.

Die Wahlhelferinnen sorgen für die Möglichkeit einer freien und geheimen Stimmabgabe, beispielsweise durch das Aufstellen von Wahlkabinen.

Beim Einwurf der Stimmzettel markieren die Wahlhelferinnen im Urnenbuch sowie im Wählerverzeichnis die Wahlen, an denen die Wählerin teilgenommen hat.

Im Umkreis von zehn Metern um Wahlurnen ist jede Beeinflussung der Wählerinnen untersagt; es dürfen nur vom Wahlausschuss genehmigte Informationen ausgelegt werden.

\Paragraph{title={ Ende der Wahl, Auszählung }}\label{wahl:auszaehlung}
Die Urnen und Urnenbücher sind nach Ende des Abstimmungszeitraums unverzüglich dem Wahlausschuss zu übergeben.

Die Auszählung soll direkt nach Ende des Abstimmungszeitraums spätestens aber am nächsten Werktag stattfinden. 

Die Auszählung findet öffentlich für Mitglieder der Studierendenschaft statt.

Der Wahlausschuss weist die Auszählhelferinnen ein und überwacht die Auszählung.

Jede Urne wird von mindestens vier Auszählhelferinnen gezählt. Sind unter den Auszählhelferinnen Kandidatinnen, so müssen diese von unterschiedlichen Wahlvorschlägen stammen.

Die Stimmzettel werden der Urne entnommen und gezählt. Ihre Zahl muss mit der Summe der Vermerke im Urnenbuch übereinstimmen. Ergibt sich nach wiederholter Zählung keine Übereinstimmung, so ist dies in der Niederschrift zu vermerken und, soweit möglich, zu erläutern. 

Die Stimmzettel werden auf ihre Gültigkeit überprüft. Ungültige Stimmzettel werden getrennt aufbewahrt und bei der Ermittlung des Abstimmungsergebnisses nicht berücksichtigt. 

Ungültig und bei der Ermittlung der Wahlergebnisse nicht anzurechnen sind Stimmzettel,
\begin{enumerate}
    \item die in Inhalt, Form und Farbe von den bereitgestellten abweichen,
    \item die ganz durchgestrichen oder ganz durchgerissen sind,
    \item die mit Bemerkungen versehen sind, ein auf die Person des Wählenden hinweisendes Merkmal oder einen Vorbehalt enthalten,
    \item aus dem sich der Wille der Wählerin nicht zweifelsfrei ergibt,
    \item deren Stimmverteilung nicht den Vorgaben gemäß \ref{wahl:wahlsystem} entspricht; werden bei der Wahl zum Studierendenparlament entweder zuviele Listen- oder zuviele Personenstimmen abgegeben, so werden nur die jeweils zuviel abgegebenen Listen- oder Personenstimmen für ungültig erklärt; die Gültigkeit der ordnungsgemäß abgegebenen Listen- oder Personenstimmen bleibt davon unberührt.
\end{enumerate}

Für jede Urne wird eine Niederschrift angefertigt. Diese enthält
\begin{enumerate}
    \item für jede Wahl einzeln die Zahl der gültigen und ungültigen Stimmzettel,
    \item für jede Wahl die auf die einzelnen Kandidatinnen entfallenen Stimmen,
    \item für die Wahl zum Studierendenparlament die auf die einzelnen Wahlvorschläge entfallenen Listenstimmen sowie die Enthaltungen bei den Listenstimmen,
    \item die Namen sowie die Unterschriften der Auszählungshelferinnen.
\end{enumerate}

\Paragraph{title={ Verteilung der Sitze und Mandate bei der Wahl zum Studierendenparlament }}\label{wahl:sitzverteilungstupa}
Bei der Wahl zum Studierendenparlament werden die auf die einzelnen Wahlvorschläge entfallenen Sitze nach dem Sainte-Laguë-Verfahren verteilt. Haben mehrere Listen die gleiche 25. Höchstzahl so entscheidet das von der Vorsitzenden des Wahlausschusses zu ziehende Los.

Die bei der Wahl zum Studierendenparlament auf die einzelnen Wahlvorschläge entfallenen Mandate werden den in den Wahlvorschlägen aufgeführten Kandidatinnen in der Reihenfolge der von ihnen erreichten Stimmenzahlen zugeteilt. Bei Stimmgleichheit entscheidet die Reihenfolge der Benennung auf dem Wahlvorschlag.

Kandidatinnen, auf die kein Mandat entfällt, sind in der Reihenfolge der von ihnen erreichten Stimmenzahlen als Ersatzleute ihres Wahlvorschlags festzustellen. 

Entfallen bei der Wahl zum Studierendenparlament auf einen Wahlvorschlag mehr Sitze als Kandidatinnen vorhanden sind, so bleiben die überzähligen Sitze unbesetzt.

\Paragraph{title={ Besetzung der Fachschaftsvorstände }}\label{wahl:fsvorstaende}
Die Anzahl der Fachschaftssprecherinnen wird gemäß \ref{fs:vorstand:anzahl} der Organisationssatzung durch die Fachschaftsordnungen geregelt. Die Kandidatinnen mit den meisten Stimmen werden Fachschaftssprecherinnen und bilden den Fachschaftsvorstand. 

Kandidatinnen, welche nicht Teil des Fachschaftsvorstands werden, werden Ersatzleute in der Reihenfolge der von ihnen erziehlten Stimmenanzahl. 

Bei Stimmgleichheit entscheidet die Reihenfolge der Benennung auf dem Wahlvorschlag.

\Paragraph{title={Feststellung des Wahlergebnisses, Wahlniederschrift }}\label{wahl:wahlniederschrift}
Der Wahlausschuss fertigt eine Wahlniederschrift an. Diese hat insbesondere zu enthalten: 
\begin{enumerate}
    \item die Namen seiner Mitglieder,
    \item den Wahlzeitraum,
    \item Vermerke über gefasste Beschlüsse,
    \item die Beschlüsse und deren Begründungen über die Ablehnung von Wahlvorschlägen oder Kandidatinnen,
    \item die Gesamtzahl der Wahlberechtigten,
    \item für jede Wahl die Zahl der insgesamt abgegebenen gültigen und ungültigen Stimmzettel,
    \item für jede Wahl die Gesamtzahl der gültigen und ungültigen Stimmen, der Enthaltungen sowie die auf die einzelnen Wahlvorschläge bzw. Kandidatinnen entfallenen Stimmen
    \item die Verteilung der Mandate auf die einzelnen Kandidatinnen und die Feststellung der Ersatzleute,
    \item die Unterschriften aller Mitglieder des Wahlausschusses.
\end{enumerate}

Mit der Unterzeichnung der Wahlniederschrift ist das Ergebnis festgestellt.

Im Anschluss an die Feststellung des Wahlergebnisses übergibt der Wahlausschuss dem Ältestenrat alle entstandenen Wahlunterlagen. Dieser hat die Wahlunterlagen zwei Monate lang aufzubewahren und dann zu vernichten. Die Vernichtung der Wahlunterlagen wird ausgesetzt, solange der Ältestenrat noch nicht über eine Anfechtung der Wahl entschieden hat.

\Paragraph{title={Bekanntmachung des Wahlergebnisses }}\label{wahl:ergebnis}
Nach der Feststellung gibt der Wahlausschuss das Wahlergebnis bekannt. Die Bekanntmachung enthält: 
\begin{enumerate}
    \item die Zahl der Wahlberechtigten,
    \item die Zahl der Wählerinnen,
    \item für jede Wahl die Gesamtzahl der gültigen und ungültigen Stimmzettel,
    \item für jede Wahl die Gesamtzahl der gültigen und ungültigen Stimmen sowie die auf die einzelnen Wahlvorschläge bzw. Kandidatinnen entfallenen Stimmen,
    \item für jede Wahl den Prozentsatz der Wahlbeteiligung,
    \item bei der Wahl zum Studierendenparlament: die auf die einzelnen Wahlvorschläge entfallenen Mandate,
    \item bei der Wahl zu den Fachschaftsvorständen: die Zusammensetzung des Fachschaftsvorstands.
\end{enumerate}

Der Wahlausschuss benachrichtigt mit der Bekanntmachung des Wahlergebnisses die gewählten Kandidatinnen. 

\Paragraph{title={Wahlanfechtung}}\label{wahl:wahlanfechtung}
Jedes Mitglied der Studierendenschaft kann die Wahl nach Maßgabe des \ref{grundsaetze:wahlen:wahlanfechtung} der Organisationssatzung anfechten.

\Paragraph{title={Berechnung der Fristen}}\label{wahl:fristen}
Bei der Berechnung der Fristen werden nur Tage gezählt, die in der vom KIT-Senat beschlossenen Vorlesungszeit liegen.

Fällt der letzte Tag einer Frist auf einen vorlesungsfreien Tag, so tritt an dessen Stelle der vorherige Vorlesungstag.

\end{jurdoc}
