\documentclass[a4paper,parskip=half,numbers=noenddot,twocolumn,titlepage,twoside]{scrartcl}

\synctex=1

%%% rubber: module xelatex
\usepackage{ifxetex}
\ifxetex
	\usepackage{fontspec}% provides font selecting commands 
	%\setmainfont[Mapping=tex-text]{Adobe Garamond Pro}
	%\setsansfont[Mapping=tex-text]{Myriad Pro}
\else
	\usepackage[T1]{fontenc}
	\usepackage[utf8]{inputenc}
\fi

\usepackage[ngerman]{babel}

% Wir brauchen scrjura aus den koma-script-Klassen in mindestens Version 0.7
% sonst werden auch einzelne Absätze nummeriert. Falls nicht vorhanden unter
% http://mirrors.ctan.org/install/macros/latex/contrib/koma-script.tds.zip
% runterladen und in ~/texmf/ entpacken (evtl. noch "texhash ~/texmf/" ausführen)
\usepackage[juratotoc,juratocindent=0pt,ref=parlong,ref=nosentence]{scrjura}[2013/11/04]


% Setze Abstand zwischen Spalten
\setlength{\columnsep}{5mm}

\usepackage{geometry}
%\geometry{left=26mm, right=25mm, top=23mm, bottom=33mm}
\geometry{
	includeheadfoot,
	margin=2.54cm
}

% Setze Titel des Dokuments und Abschnitt (wenn vorhanden) als Kopfzeile
\usepackage{scrlayer-scrpage}
\pagestyle{scrheadings}
\automark[part]{part}
\automark*[section]{}

\renewcommand{\thesection}{\Alph{section}}

% Setze einen kleinen Abstand \, zwischen Zahl und Buchstabe bei Paragraphen
\renewcommand*{\thecontractSubParagraph}{%
{\theParagraph\,\alph{contractSubParagraph})}}

% Größerer Abstand zwischen Paragraphennummer und -titel im Inhaltsverzeichnis
\renewcommand{\numberline}[1]{\makebox[2.5em][l]{#1}}

% Parts stehen bei scrartcl nicht auf einer eigenen Seite => Fix
\renewcommand\partheadstartvskip{\cleardoublepage%
	\thispagestyle{empty}%
	\onecolumn%
	\null\vfil%
}
\renewcommand\partheadmidvskip{\par\nobreak\vskip 20pt}
\renewcommand\partheadendvskip{\vfil\newpage\twocolumn}
\renewcommand\raggedpart{\centering}

% Aus irgendeinem Grund werden chapters nicht als Absatz in einem Paragraphen
% fehlinterpretiert sections aber schon und dementsprechend wird vor ihnen eine
% Absatznummer eingefügt => definiere eigenen Befehl (macht es auch einfacher
% wenn man doch wieder scrbook und chapters will)
\newcommand{\jursection}[1]{\parnumberfalse\section{#1}\parnumbertrue}

\title{Organisationssatzung \\ der Verfassten Studierendenschaft \\ des Karlsruher Instituts für Technologie (KIT)}
\author{}
\date{}

\usepackage[pdfusetitle, pdfborder={0 0 0}]{hyperref}

% Sections sollten für jedes Dokument einzeln gezählt werden. Muss nach hyperref
% passieren sonst sind die Links falsch.
\makeatletter
\@addtoreset{section}{part}
\makeatother

\begin{document}



\maketitle

% tableofcontents von scrartcl beginnt direkt => füge Vakatseiten ein
\cleardoublepage

\tableofcontents

Aufgrund von § 65 a Absatz 1 des Gesetzes über die Hochschulen in Baden-Württemberg (Landeshochschulgesetz -- LHG) in der Fassung vom 01. Januar 2005 (GBl. S. 1 f.), zuletzt geändert durch Artikel 2 des Gesetzes zur Einführung einer Verfassten Studierendenschaft und zur Stärkung der akademischen Weiterbildung (Verfasste-Studierendenschafts-Gesetz -- VerfStudG) in der Fassung vom 10. Juli 2012 (GBl. S. 457 ff.) hat die Studierendenschaft des KIT die nachstehende Organisationssatzung beschlossen.

Im Folgenden wird aus Gründen der besseren Lesbarkeit ausschließlich die weibliche Form verwendet. Dabei ist jede andere Form impliziert. Die Geschlechtsdefinition obliegt jeder Person selbst.




\part[Organisationssatzung]{Organisationssatzung \\ der Verfassten Studierendenschaft \\ des Karlsruher Instituts für Technologie (KIT)}

\input{vs-satzung.core.tex}


\part[Wahl- und Abstimmungsordnung]{Wahl- und Abstimmungsordnung \\ der Verfassten Studierendenschaft \\ des Karlsruher Instituts für Technologie (KIT)}

\input{wahl-abstimmungsordnung.core.tex}


\part{Übergangsbestimmungen}

\input{uebergangsbestimmungen.core.tex}

%\fbox{
%\begin{tabular}{ll}
%Impressum & \\
%\small Herausgeber: & UStA KIT \\
%Tel.: & 0721/608-48460 \\
%Web: & www.usta.de \\
%E-Mail: & info$@$usta.de\\
%V.i.S.d.P. & Philipp Rudo\\
%Druck: & Studierenden Service \\
%& Verein (SSV)\\
%Auflage: & 300 Exemplare
%\end{tabular}
%}

\end{document}
