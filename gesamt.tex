\documentclass[a4paper,parskip=half,numbers=noenddot,titlepage,twoside]{scrartcl}

\synctex=1

%%% rubber: module xelatex
\usepackage{ifxetex}
\ifxetex
	\usepackage{fontspec}% provides font selecting commands 
	%\setmainfont[Mapping=tex-text]{Adobe Garamond Pro}
	%\setsansfont[Mapping=tex-text]{Myriad Pro}
\else
	\usepackage[T1]{fontenc}
	\usepackage[utf8]{inputenc}
\fi

\usepackage[ngerman]{babel}

% Wir brauchen scrjura aus den koma-script-Klassen in mindestens Version 0.7
% sonst werden auch einzelne Absätze nummeriert. Falls nicht vorhanden unter
% http://mirrors.ctan.org/install/macros/latex/contrib/koma-script.tds.zip
% runterladen und in ~/texmf/ entpacken (evtl. noch "texhash ~/texmf/" ausführen)
\usepackage[juratotoc,juratocindent=0pt,ref=parlong,ref=nosentence]{scrjura}[2013/11/04]

% Verwende multicol für den Zwei-Spalten-Satz
% Erlaubt Umschaltung innerhalb der Seite und gleicht die Spalten aus
\usepackage{multicol}
% Setze Abstand zwischen Spalten
\setlength{\columnsep}{5mm}

\usepackage{geometry}
%\geometry{left=26mm, right=25mm, top=23mm, bottom=33mm}
\geometry{
	includeheadfoot,
	margin=2.54cm
}

% Setze Titel des Dokuments und Abschnitt (wenn vorhanden) als Kopfzeile
\usepackage{scrlayer-scrpage}
\pagestyle{scrheadings}
\automark[part]{part}
\automark*[section]{}

\renewcommand{\thesection}{\Alph{section}}

% Setze einen kleinen Abstand \, zwischen Zahl und Buchstabe bei Paragraphen
\renewcommand*{\thecontractSubParagraph}{%
{\theParagraph\,\alph{contractSubParagraph})}}

% Größerer Abstand zwischen Paragraphennummer und -titel im Inhaltsverzeichnis
\renewcommand{\numberline}[1]{\makebox[2.5em][l]{#1}}

% Parts stehen bei scrartcl nicht auf einer eigenen Seite => Fix
\renewcommand\partheadstartvskip{\cleardoublepage%
	\thispagestyle{empty}%
	\null\vfil%
}
\renewcommand\partheadmidvskip{\par\nobreak\vskip 20pt}
\renewcommand\partheadendvskip{\vfil\newpage}
\renewcommand\raggedpart{\centering}

% Aus irgendeinem Grund werden chapters nicht als Absatz in einem Paragraphen
% fehlinterpretiert sections aber schon und dementsprechend wird vor ihnen eine
% Absatznummer eingefügt => definiere eigenen Befehl (macht es auch einfacher
% wenn man doch wieder scrbook und chapters will)
\newcommand{\jursection}[1]{\parnumberfalse\section{#1}\parnumbertrue}

\title{Organisationssatzung \\ der Verfassten Studierendenschaft \\ des Karlsruher Instituts für Technologie (KIT)}
\author{}
\date{}

\usepackage[pdfusetitle, pdfborder={0 0 0}]{hyperref}

% Sections sollten für jedes Dokument einzeln gezählt werden. Muss nach hyperref
% passieren sonst sind die Links falsch.
\makeatletter
\@addtoreset{section}{part}
\makeatother

\begin{document}



\maketitle

% tableofcontents von scrartcl beginnt direkt => füge Vakatseiten ein
\cleardoublepage

\begin{multicols}{2}
\tableofcontents
\end{multicols}

\bigskip

Aufgrund von § 65 a Absatz 1 des Gesetzes über die Hochschulen in Baden-Württemberg (Landeshochschulgesetz -- LHG) in der Fassung vom 01. Januar 2005 (GBl. S. 1 f.), zuletzt geändert durch Artikel 2 des Gesetzes zur Einführung einer Verfassten Studierendenschaft und zur Stärkung der akademischen Weiterbildung (Verfasste-Studierendenschafts-Gesetz -- VerfStudG) in der Fassung vom 10. Juli 2012 (GBl. S. 457 ff.) hat die Studierendenschaft des KIT die nachstehende Organisationssatzung beschlossen.

Im Folgenden wird aus Gründen der besseren Lesbarkeit ausschließlich die weibliche Form verwendet. Dabei ist jede andere Form impliziert. Die Geschlechtsdefinition obliegt jeder Person selbst.




\part[Organisationssatzung]{Organisationssatzung \\ der Verfassten Studierendenschaft \\ des Karlsruher Instituts für Technologie (KIT)}

\begin{multicols}{2}
%
% STUDIERENDENSCHAFT
%

\begin{contract}
\setcounter{Paragraph}{0}


\parnumberfalse \section{Studierendenschaft} \parnumbertrue

\Paragraph{title=Studierendenschaft}
\parnumberfalse Die immatrikulierten Studierenden einschließlich der immatrikulierten Doktorandinnen des Karlsruher Instituts für Technologie (KIT) (Mitglieder) bilden gemäß § 65 Absatz 1 LHG die Studierendenschaft. Sie ist eine rechtsfähige Körperschaft des öffentlichen Rechts und als solche eine Gliedkörperschaft des KIT. Die Studierendenschaft arbeitet auf demokratischer Grundlage und wahrt nach den verfassungsrechtlichen Grundsätzen die weltanschauliche, religiöse und parteipolitische Neutralität.\parnumbertrue

\Paragraph{ title = Aufgaben}\label{studierendenschaft:aufgaben}

\parnumberfalse Die Studierendenschaft verwaltet ihre Angelegenheiten im Rahmen der gesetzlichen Bestimmungen selbst. Sie hat unbeschadet der Zuständigkeit der Hochschule und des Studentenwerks die folgenden Aufgaben
\begin{enumerate}
\item die Wahrnehmung der hochschulpolitischen, fachlichen und fachübergreifenden sowie der sozialen, wirtschaftlichen und kulturellen Belange der Studierenden,
\item die Mitwirkung an den Aufgaben des KIT nach §§~2~bis~7~LHG i.~V.~m. §~20~KITG,
\item die Förderung der politischen Bildung und des staatsbürgerlichen Verantwortungsbewusstseins der Studierenden,
\item die Förderung der Gleichstellung und den Abbau von Benachteiligungen innerhalb der Studierendenschaft,
\item die Förderung der sportlichen und musischen Aktivitäten der Studierenden,
\item die Pflege und der Ausbau der überregionalen und internationalen Studierendenbeziehungen.
\end{enumerate}\parnumbertrue

\Paragraph{title={Rechte und Pflichten der Mitglieder}}\label{studierendenschaft:mitglieder}

Jedes Mitglied hat das aktive Wahlrecht.

Soweit diese Satzung keine Einschränkungen vorsieht, hat jedes Mitglied das passive Wahlrecht.

Jeweils 25 Mitglieder haben das Recht, Anfragen an die Organe nach \ref{organe:organe} Nummer \ref{organe:organe:stupa}, \ref{organe:organe:vorstand}, \ref{organe:organe:aera} und \ref{organe:organe:fsk} zu stellen. Anfragen sind schriftlich an die Vorsitzende des betreffenden Organs zu richten. 
Anfragen müssen vom Vorstand innerhalb von vier Wochen und von den anderen Organen innerhalb von vier Wochen während der Vorlesungszeit in Textform beantwortet werden.\label{studierendenschaft:mitglieder:anfragen}

Jeweils 25 Mitglieder haben ein Antragsrecht an die Organe nach \ref{organe:organe} Nummer \ref{organe:organe:stupa}, \ref{organe:organe:vorstand}, \ref{organe:organe:aera} und \ref{organe:organe:fsk}. Anträge sind schriftlich an die Vorsitzende des betreffenden Organs zu richten.\label{studierendenschaft:mitglieder:antraege}

Jedes Mitglied hat das Recht der Beschwerde gegen Maßnahmen und Beschlüsse der Organe der Studierendenschaft, insbesondere mit der Behauptung, es liege ein Verstoß gegen die Organisationssatzung vor. Beschwerden sind schriftlich an den Ältestenrat zu richten. \label{studierendenschaft:mitglieder:beschwerden} 

\Paragraph{ title = Organe der Studierendenschaft}

Die Organe der Studierendenschaft auf zentraler Ebene sind \label{organe:organe}
\begin{enumerate}
\item die Vollversammlung,
\item das Studierendenparlament, \label{organe:organe:stupa}
\item der Vorstand, \label{organe:organe:vorstand}
\item der erweiterte Vorstand,
\item der Ältestenrat, \label{organe:organe:aera}
\item die Fachschaftenkonferenz. \label{organe:organe:fsk}
\end{enumerate}

Die Organe der Studierendenschaft tagen grundsätzlich öffentlich. Die Geschäftsordnung des jeweiligen Organs kann in begründeten Fällen -- insbesondere in Personalangelegenheiten und aus datenschutzrechtlichen Gründen -- Ausnahmen hiervon vorsehen.

Über die Sitzungen der Organe sind Protokolle anzufertigen; diese müssen veröffentlicht werden. Näheres regelt die Geschäftsordnung des jeweiligen Organs.

%
% URABSTIMMUNG
%

\parnumberfalse \section{Urabstimmung}\label{urabstimmung} \parnumbertrue

\Paragraph{title=Aufgaben}
\parnumberfalse Die Urabstimmung entscheidet über grundlegende Fragen der Studierendenschaft. Sie kann über die Auflösung des Studierendenparlaments sowie über Änderungen der Organisationssatzung entscheiden.\parnumbertrue

\Paragraph{title=Stimmrecht}
\parnumberfalse Jedes Mitglied ist bei der Urabstimmung stimmberechtigt.\parnumbertrue

\Paragraph{title=Zustandekommen}\label{urabstimmung:zustandekommen}
\parnumberfalse Die Urabstimmung findet statt
\begin{enumerate}
\item auf Beschluss des Studierendenparlaments,
\item auf Beschluss der Fachschaftenkonferenz mit mehr als der Hälfte der satzungsgemäß existierenden Stimmen, \label{urabstimmung:zustandekommen:fsk}
\item auf Antrag der Mitglieder; zu diesem Zweck muss ein entsprechender Antrag von mindestens 5~\% der Mitglieder unterzeichnet sein; der Antrag ist schriftlich beim Ältestenrat einzureichen; dieser beantragt nach Überprüfung der Voraussetzungen unverzüglich eine Sitzung des Studierendenparlaments zur Wahl des Wahlausschusses. \label{urabstimmung:zustandekommen:mitglieder}
\end{enumerate} \parnumbertrue

\Paragraph{title=Organisation und Ablauf}\label{urabstimmung:orga}

Findet gemäß \ref{urabstimmung:zustandekommen} eine Urabstimmung statt, so wählt das Studierendenparlament unverzüglich einen Wahlausschuss für die Durchführung der Urabstimmung; dazu ist der Beschluss nach \ref{urabstimmung:zustandekommen} Nummer \ref{urabstimmung:zustandekommen:fsk} bzw. die Erfüllung  der Bedingungen nach \ref{urabstimmung:zustandekommen}  Nummer \ref{urabstimmung:zustandekommen:mitglieder} dem Präsidium des Studierendenparlaments unverzüglich mitzuteilen.

Es gelten die Vorschriften des \ref{grundsaetze:wahlen}. Darüber hinaus regelt die Wahl- und Abstimmungsordnung weitere Einzelheiten.


\Paragraph{title=Beschlüsse}

Beschlüsse der Urabstimmung sind gültig und bindend für die Organe der Studierendenschaft, wenn mindestens ein Sechstel aller Mitglieder sowie die Mehrheit der an der Abstimmung teilnehmenden Mitglieder zugestimmt haben. 

Die Urabstimmung entscheidet bei Änderungen der Organisationssatzung mit Zweidrittelmehrheit der abgegebenen Stimmen. 

Beschlüsse der Urabstimmung heben widersprechende Beschlüsse der Vollversammlung und des Studierendenparlaments auf.

%
% VOLLVERSAMMLUNG
%

\parnumberfalse \section{Vollversammlung} \parnumbertrue

\Paragraph{title=Aufgaben}

Die Vollversammlung ist ein beschließendes Organ der Studierendenschaft und dient der Information der Mitglieder. 

Die Vollversammlung kann nicht über Änderungen der Organisationssatzung sowie Erlass und Änderung weiterer Satzungen, der Finanzordnung, der Beitragsordnung und des Haushalts- oder Wirtschaftsplans beschließen.

\Paragraph{title=Stimm- und Antragsrecht}

\parnumberfalse Jedes Mitglied ist auf der Vollversammlung stimm- und antragsberechtigt.\parnumbertrue

\Paragraph{title=Zustandekommen} \label{vv:zustandekommen}

\parnumberfalse Eine Vollversammlung findet statt
\begin{enumerate}
\item auf Beschluss des Studierendenparlaments,
\item auf Beschluss der Fachschaftenkonferenz mit mehr als der Hälfte der satzungsgemäß existierenden Stimmen,
\item \label{vv:zustandekommen:mitglieder}auf Antrag der Mitglieder; zu diesem Zweck muss ein entsprechender Antrag von mindestens 2~\% der Mitglieder unterzeichnet sein; der Antrag ist schriftlich beim Ältestenrat einzureichen.
\end{enumerate} \parnumbertrue


\Paragraph{title=Organisation und Ablauf}\label{vv:organisation}

Die Organisation der Vollversammlung obliegt dem Ältestenrat; er kann den Vorstand damit beauftragen.

Die Vollversammlung findet spätestens 30~Tage nach dem Beschluss des Studierendenparlaments oder der Fachschaftenkonferenz bzw. dem Eingang des Antrags der Mitglieder statt, sofern im Beschluss oder Antrag kein Zeitpunkt genannt ist oder der genannte Zeitpunkt die rechtzeitige Einladung nicht zulässt.

Die Einladung zur Vollversammlung erfolgt durch eine Bekanntmachung entsprechend \ref{grundsaetze:wahlen:bekanntmachung} mit einer Frist von einer Woche. Die Bekanntmachung enthält einen Vorschlag für die Tagesordnung, der alle auf Einberufungsanträgen gewünschten Tagesordnungspunkte enthalten muss.

Vollversammlungen sind öffentlich. Die anwesenden Mitglieder haben Rederecht. Nichtmitglieder können auf Antrag von der Vollversammlung ausgeschlossen werden.

Zu Beginn der Vollversammlung wird ein Präsidium gewählt. Der Ältestenrat macht hierzu einen Vorschlag. Dem Präsidium darf kein Mitglied des Ältestenrates angehören. Das Präsidium besteht aus einer Präsidentin und drei Stellvertreterinnen. Das Präsidium ist für die ordnungsgemäße Durchführung der Vollversammlung verantwortlich.

Das Protokoll der Vollversammlung ist binnen einer Woche fertigzustellen und dem Studierendenparlament vorzulegen.

Das Studierendenparlament kann eine Geschäftsordnung für die Vollversammlung beschließen. Ist eine solche nicht vorhanden, so findet die Ge\-schäfts\-ord\-nung des Studierendenparlaments sinngemäß Anwendung. Die Vollversammlung kann Abweichungen von der Geschäftsordnung beschließen.



\Paragraph{title=Beschlüsse}

Beschlüsse der Vollversammlung sind gültig und wirksam, wenn mindestens 5~\% aller Mitglieder sowie die Mehrheit der an der Abstimmung teilnehmenden Mitglieder zugestimmt haben. Erreicht ein Beschluss dieses Quorum nicht, so behandelt das Studierendenparlament diesen auf seiner nächsten Sitzung.

Beschlüsse der Vollversammlung heben widersprechende Beschlüsse des Studierendenparlaments auf.

Beschlüsse der Vollversammlung sind ausgesetzt, sobald eine Urabstimmung dazu beantragt ist.


%
% STUDIERENDENPARLAMENT
%

\parnumberfalse \section{Studierendenparlament} \parnumbertrue

\Paragraph{title=Aufgaben}
Das Studierendenparlament ist das beschließende Organ der Studierendenschaft; es ist das legislative Organ gemäß §~65~a Absatz~3 Satz~2~LHG.

Das Studierendenparlament ist insbesondere zuständig für
\begin{enumerate}
\item die Wahl und Abberufung der Vorstandsmitglieder,
\item die Wahl des Ältestenrats,
\item Änderungen der Organisationssatzung,
\item den Beschluss sonstiger Satzungen,
\item den Beschluss über den Haushalt der Studierendenschaft,
\item die Entscheidung über die Führung eines Wirtschaftsplans (§~110~LHO) anstelle eines Haushaltsplans (§~106~LHO),
\item den Beschluss über alle sonstigen Maßnahmen, die die Studierendenschaft langfristig finanziell belasten,
\item den Zusammenschluss mit studentischen Vertretungen anderer Hochschulen,
\item die Wahl des Wahlausschusses,
\item die Wahl von Vertreterinnen in den Finanzausschuss nach \ref{haushalt:finanzausschuss:wahl},
\item die Wahl von studentischen Mitgliedern in Gremien auf zentraler Ebene des KIT, soweit hierzu keine direkten Wahlen stattfinden.
\end{enumerate}


\Paragraph{title={Zusammensetzung, Wahl}}\label{stupa:zusammensetzung}

Das Studierendenparlament besteht aus 25 Abgeordneten, die von den Mitgliedern der Studierendenschaft nach den Grundsätzen der Verhältniswahl allgemein, gleich, frei, geheim und unmittelbar gewählt werden. Es gelten die Vorschriften des \ref{grundsaetze:wahlen}. Darüber hinaus regelt die Wahl- und Abstimmungsordnung weitere Einzelheiten.

Eine Abgeordnete scheidet aus \label{stupa:zusammensetzung:ausscheiden} 
  \begin{enumerate}
  \item am Ende der Amtsperiode,
  \item durch Exmatrikulation,
  \item durch eigenen Verzicht; dieser ist dem Präsidium des Studierendenparlaments in Textform mitzuteilen,
  \item bei Auflösung des Studierendenparlaments,
  \item durch automatischen Ausschluss bei dreimaligem unentschuldigtem Fehlen bzw. bei insgesamt fünfmaliger Abwesenheit von den Sitzungen des Studierendenparlaments; die Feststellung erfolgt durch das Präsidium des Studierendenparlaments; näheres regelt die Geschäftsordnung; liegen triftige Gründe für das Fehlen vor, kann der Ältestenrat innerhalb von 14 Tagen die Wiederanerkennung des Sitzes verfügen; nachgerückte Abgeordnete verlieren in diesem Falle wieder ihren Sitz. \label{stupa:zusammensetzung:ausscheiden:wiederanerkennung}
\end{enumerate}
Bei Ausscheiden einer Abgeordneten rückt die Nächste auf der Liste nach. Ist die Liste erschöpft, so bleibt der Sitz unbesetzt.

Die Amtsperiode des Studierendenparlaments beginnt in der Regel am 1. Oktober und endet am darauffolgenden 30. September.


\Paragraph{title={Organisation und Ablauf}}

Das Studierendenparlament gibt sich eine Geschäftsordnung.

Das Studierendenparlament wählt sich in jeder Amtsperiode aus seiner Mitte ein Präsidium. Das Präsidium besteht aus einer Präsidentin und zwei Stellvertreterinnen. Das Präsidium ist für die ordnungsgemäße Einberufung und Durchführung der Sitzungen verantwortlich. Seine Mitglieder haben in der Studierendenschaft uneingeschränktes Informationsrecht.

Antragsberechtigt in Sitzungen des Studierendenparlaments sind
  \begin{enumerate}
  \item die Abgeordneten,
  \item die Mitglieder des Vorstandes der Studierendenschaft,
  \item der Ältestenrat,
  \item die Fachschaftsvorstände,
  \item die Fachschaftenkonferenz,
  \item die Präsidentin der Fachschaftenkonferenz,
  \item die Mitglieder nach Maßgabe von \ref{studierendenschaft:mitglieder:antraege}.
  \end{enumerate}

Das Studierendenparlament tagt mindestens einmal pro Vorlesungsmonat. Darüber hinaus muss es auf Antrag des Vorstands, des Ältestenrats oder eines Viertels der Abgeordneten einberufen werden.

Das Studierendenparlament wird von der Präsidentin in Textform einberufen. Mit der Einberufung ist die vorgeschlagene Tagesordnung bekanntzumachen.

Die Abgeordneten sind verpflichtet, an jeder Sitzung persönlich teilzunehmen. Das Stimmrecht kann nicht delegiert werden. Entschuldigungen sind beim Präsidium vor der Sitzung in Textform einzureichen.

Die Abgeordneten haben das Recht, Anfragen an den Vorstand zu stellen. Anfragen sind schriftlich an die zuständige Referentin zu richten und müssen innerhalb von vier Wochen in Textform beantwortet werden.

Die Abgeordneten haben das Recht, Einsicht in die Unterlagen des Vorstands zu verlangen. Der Vorstand hat das Verlangen binnen zwei Wochen zu erfüllen, indem er die Unterlagen in seinen Räumen zur Einsicht vorlegt. Enthalten die Unterlagen personenbezogene Daten, so Bedarf die Einsicht der Zustimmung der betroffenen Personen.


\Paragraph{title=Beschlüsse}\label{stupa:beschluesse}

Das Studierendenparlament ist beschlussfähig, wenn mehr als die Hälfte der Mitglieder des Studierendenparlaments anwesend sind. Wird zu Beginn oder während der Sitzung festgestellt, dass das Studierendenparlament nicht beschlussfähig ist, so wird die Sitzung vertagt. Das Studierendenparlament ist auf der nächsten Sitzung in Bezug auf die vertagten Punkte, unbeschadet \refPar{stupa:beschluesse:zweidrittel}, beschlussfähig.

  Für folgende Beschlüsse ist eine Zweidrittelmehrheit der Stimmberechtigten des Studierendenparlaments erforderlich \label{stupa:beschluesse:zweidrittel} 
  \begin{enumerate}
  \item Selbstauflösung des Studierendenparlaments,
  \item Änderung der Organisationssatzung oder der Erlass bzw. Änderung weiterer Satzungen sowie der Ge\-schäfts\-ord\-nungen von Studierendenparlament und Vollversammlung,
  \item Änderung des Haushalts- oder Wirtschaftsplans,
  \item Aufhebung eines Vetos der Fachschaftenkonferenz nach \ref{fsk:aufgaben:einspruch}.
  \end{enumerate}


%
% VORSTAND
%

\parnumberfalse \section{Vorstand} \parnumbertrue

\Paragraph{title=Aufgaben}

Der Vorstand ist das ausführende Organ der Studierendenschaft; er ist das exekutive Kollegialorgan gemäß §~65~a Absatz~3 Satz~3~LHG. 

Der Vorstand führt die laufenden Geschäfte in eigener Verantwortung im Rahmen der Beschlüsse von Studierendenparlament, Vollversammlung und Urabstimmung. Er ist dem Studierendenparlament rechenschaftspflichtig.

Der Vorstand wählt aus seiner Mitte eine Person, die mit beratender Stimme an den Sitzungen des Senats teilnimmt.

Der Vorstand vertritt die Studierendenschaft in der landesweiten Vertretung der Studierendenschaften nach §~65~a Absatz~8~LHG.

Der Vorstand kann sich eine Geschäftsordnung geben.


\Paragraph{title={Zusammensetzung, Wahl}} \label{vorstand:zusammensetzung}

Der Vorstand der Studierendenschaft besteht in der Regel aus folgenden Referaten \label{vorstand:zusammensetzung:regel}
\begin{enumerate}
\item Vorsitz,
\item Finanzen,
\item Inneres,
\item Soziales I,
\item Soziales II,
\item Äußeres,
\item Ökologie,
\item Presse und Öffentlichkeitsarbeit,
\item Kultur,
\item Chancengleichheit,
\item Ausländerinnen.
\end{enumerate}
Veränderungen dieser Struktur können vom Studierendenparlament mit absoluter Mehrheit beschlossen werden; die Referate Vorsitz, Finanzen, Chancengleichheit und Ausländerinnen bleiben hiervon unberührt. Die Anzahl der Referate darf zwölf nicht übersteigen.

Das Studierendenparlament besetzt zu Beginn seiner Amtszeit die Referate durch Wahl in getrennten Wahlgängen mit je einem Mitglied der Studierendenschaft. Einem Antrag auf geheime Wahl muss stattgegeben werden\label{vorstand:zusammensetzung:wahl}.

Der Vorstand ist im Amt, wenn Vorsitz und Finanzreferat besetzt sind.

Die Vorsitzende vertritt die Studierendenschaft. Ist die Vorsitzende  verhindert wird sie durch die Finanzreferentin vertreten, es sei denn der Vorstand hat vorher ausdrücklich eine andere Referentin bestimmt.

Die Vorstandsmitglieder scheiden aus
  \begin{enumerate}
  \item mit der Wahl eines neuen Vorstands gemäß \refPar{vorstand:zusammensetzung:wahl},
  \item durch Exmatrikulation,
  \item durch eigenen Verzicht,
  \item durch konstruktives Misstrauensvotum des Studierendenparlaments.
  \end{enumerate}
    Ist ein Referat nach \ref{vorstand:zusammensetzung:regel} nicht besetzt, führt das Studierendenparlament eine Nachwahl für den Rest der Amtszeit durch.

Ist das Chancengleichheitsreferat durch einen Mann besetzt, muss eine Frau zur Unterstützung gemäß \ref{erweitertervorstand:wahl} in den erweiterten Vorstand gewählt werden; ist es durch eine Frau besetzt, muss entsprechend ein Mann gewählt werden.


%
% ERWEITERTER VORSTAND
%

\parnumberfalse \section{Erweiterter Vorstand} \parnumbertrue

\Paragraph{title=Aufgaben}\label{erweitertervorstand:aufgaben}
\parnumberfalse Der erweiterte Vorstand unterstützt den Vorstand bei seiner Arbeit. Er ist diesem rechenschaftspflichtig.\parnumbertrue

\Paragraph{title=Wahl} \label{erweitertervorstand:wahl}

Die Mitglieder des erweiterten Vorstands werden vom Vorstand gewählt. Diese müssen vom Studierendenparlament einzeln bestätigt werden, einem Antrag auf geheime Abstimmung muss stattgegeben werden.

Die Mitglieder des erweiterten Vorstands scheiden aus
\begin{enumerate}
  \item mit der Wahl eines neuen Vorstands gemäß \ref{vorstand:zusammensetzung:wahl},
  \item durch Exmatrikulation,
  \item durch eigenen Verzicht,
  \item durch Beschluss des Vorstandes mit absoluter Mehrheit,
  \item durch Beschluss des Studierendenparlaments mit absoluter Mehrheit.
\end{enumerate}


%
% ÄLTESTENRAT
%

\parnumberfalse \section{Ältestenrat}\label{aera} \parnumbertrue

\Paragraph{title=Aufgaben}\label{aera:aufgaben}

Der Ältestenrat ist die Schlichtungskommission gemäß §~65~a Absatz~9~LHG. Darüber hinaus hat er folgende Aufgaben: \label{aera:aufgaben:allgemein}
  \begin{enumerate}
  \item Aufhebung satzungswidriger Beschlüsse gemäß \ref{studierendenschaft:mitglieder:beschwerden},
  \item \label{aera:aufgaben:vv} Organisation einer Vollversammlung gemäß \ref{vv:organisation},
  \item \label{aera:aufgaben:ua} Entgegennahme und Prüfung eines Antrags auf Urabstimmung  gemäß \ref{urabstimmung:zustandekommen} Nummer \ref{urabstimmung:zustandekommen:mitglieder} oder Vollversammlung gemäß \ref{vv:zustandekommen} Nummer \ref{vv:zustandekommen:mitglieder},
  \item Entscheidung über die Anfechtung einer Wahl oder Abstimmung gemäß \ref{grundsaetze:wahlen:wahlanfechtung},
  \item Wiederanerkennung eines Sitzes im Studierendenparlament  gemäß \ref{stupa:zusammensetzung:ausscheiden} Nummer \ref{stupa:zusammensetzung:ausscheiden:wiederanerkennung},
  \item Feststellung von Verstößen gegen die Organisationssatzung oder weiterer Satzungen,
  \item Prüfung der Fachschaftsordnungen.
  \end{enumerate}

Der Ältestenrat tagt mindestens einmal pro Vorlesungsmonat. Die Mitglieder sind zur Teilnahme an den Sitzungen verpflichtet.

Dem Studierendenparlament sind Protokolle der Sitzungen vorzulegen. Ein Mitglied des Ältestenrats soll ihm für Rückfragen zur Verfügung stehen.

Die Mitglieder des Ältestenrates haben in der Studierendenschaft uneingeschränktes Informationsrecht.

Eingaben an den Ältestenrat sind an die Vorsitzende zu richten. Sie versieht die Eingabe mit dem Eingangsdatum und veranlasst die Behandlung in der nächsten Sitzung. Über das Ergebnis ist die Eingebende zu unterrichten.

Ist der Ältestenrat mit zwei oder weniger Mitgliedern besetzt, so übernimmt das Präsidium des Studierendenparlaments im Einvernehmen mit den amtierenden Mitgliedern des Ältestenrats dessen Aufgaben nach \refPar{aera:aufgaben:allgemein} Nummer \ref{aera:aufgaben:vv} und \ref{aera:aufgaben:ua}.


\Paragraph{title=Zusammensetzung}

Der Ältestenrat besteht aus fünf Mitgliedern. Sie werden vom Studierendenparlament auf ein Jahr gewählt. Die Amtszeiten der einzelnen Mitglieder beginnen entweder am 1. April oder 1. Oktober; sie sollen nicht alle am gleichen Datum beginnen.

Die Mitglieder des Ältestenrats sollen ehemalige Mitglieder der studentischen Selbstverwaltung sein.

Die Mitglieder des Ältestenrats dürfen weder Mitglieder eines anderen Organs der Studierendenschaft noch eines Organs des KIT sein oder für eines kandidieren.

Mitglieder des Ältestenrats scheiden aus
  \begin{enumerate}
  \item am Ende ihrer Amtszeit,
  \item durch Exmatrikulation,
  \item durch eigenen Verzicht,
  \item durch automatischen Ausschluss bei dreimaligem unentschuldigtem Fehlen bzw. bei insgesamt fünfmaliger Abwesenheit. 
  \end{enumerate}
Bei vorzeitigem Ausscheiden eines Mitglieds erfolgt eine Nachwahl durch das Studierendenparlament für den Rest der Amtszeit.


\Paragraph{title=Organisation}

Der Ältestenrat wählt sich seine Vorsitzende aus seiner Mitte.

Das Studierendenparlament kann auf Vorschlag des Ältestenrats eine Geschäftsordnung für den Ältestenrat beschließen. Ist eine solche nicht vorhanden, so findet die Ge\-schäfts\-ord\-nung des Studierendenparlaments sinngemäß Anwendung.


\Paragraph{title=Beschlüsse}

Erklärt der Ältestenrat einen Beschluss eines Organs der Studierendenschaft für satzungswidrig, so ist dieser aufgehoben. Die Aufhebung eines Beschlusses ist schriftlich zu begründen und dem jeweiligen Organ mitzuteilen. Ein Mitglied des Ältestenrats soll dem jeweiligen Organ für Rückfragen zur Verfügung stehen.

Erklärt der Ältestenrat die Anfechtung einer Wahl oder Abstimmung für begründet, so veranlasst er die zur Behebung des Mangels erforderlichen Tätigkeiten. Kann der Mangel nicht behoben werden, so ist die Wahl oder Abstimmung ungültig und muss wiederholt werden. 

Erhält der Ältestenrat den Antrag auf Wiederanerkennung eines Sitzes im Studierendenparlament, so gibt er der betroffenen Abgeordneten Gelegenheit zur Stellungnahme. Kann sie sich angemessen rechtfertigen, so erkennt der Ältestenrat den Sitz wieder an und teilt dies dem Präsidium des Studierendenparlaments mit.

%wann sind Beschlüsse des Ära gültig?

%
% FACHSCHAFTEN
%

\parnumberfalse \section{Fachschaften} \parnumbertrue

\Paragraph{title=Aufgaben} \label{fachschaften:aufgaben}

\parnumberfalse Die Organe der Fachschaft nehmen die fakultätsbezogenen Studienangelegenheiten und Aufgaben im Sinne des \ref{studierendenschaft:aufgaben} auf Fakultätsebene wahr. \parnumbertrue

\Paragraph{title={Gliederung, Mitgliedschaft}}\label{fachschaften:gliederung}

Die Studierenden einer Fakultät bilden eine Fachschaft. 

Die Fachschaften regeln ihre Angelegenheiten durch Fachschaftsordnungen selbst. Diese sollen dem Ältestenrat zur Prüfung vorgelegt werden. Fachschaftsordnungen sind vom Studierendenparlament als Satzungen zu beschließen.


\Paragraph{title=Organe}

Organe der Fachschaft sind
  \begin{enumerate}
  \item der Fachschaftsvorstand,
  \item die Fachschaftsversammlung.
  \end{enumerate}

Die Fachschaftsordnung kann weitere Organe vorsehen.


\Paragraph{title=Fachschaftsvorstand}\label{fs:vorstand}

Der Fachschaftsvorstand ist das ausführende Organ der Fachschaft. Näheres regelt die Fachschaftsordnung.

Der Fachschaftsvorstand besteht aus den Fach\-schafts\-sprech\-erinnen. Die Fach\-schafts\-sprech\-erinnen werden durch allgemeine, gleiche, geheime und direkte Wahl nach dem Grundsatz der Persönlichkeitswahl gewählt. Die Amtsperiode des Fachschaftsvorstandes beginnt in der Regel am 1. Oktober und endet am darauffolgenden 30. September. Es gelten die Vorschriften des \ref{grundsaetze:wahlen}. Näheres bestimmt die Wahl- und Abstimmungsordnung.

Die Anzahl der Fachschaftssprecherinnen wird unter Beachtung der Anzahl der Studierenden in der Fachschaftsordnung festgelegt. Sie beträgt mindestens zwei und höchstens acht. \label{fs:vorstand:anzahl}

%
% Beschluss FSK: bis 1000 Studis 2 FS-Sprecher, dann je angefangener 500 Studis einer mehr
%

Eine Fachschaftssprecherin scheidet aus dem Amt
  \begin{enumerate}
  \item am Ende der Amtsperiode,
  \item durch Exmatrikulation,
  \item durch eigenen Verzicht,
  \item bei Wahl eines neuen Vorstandes nach \ref{fachschaft:vv:wahl}.
\end{enumerate}

Bei Ausscheiden einer Fach\-schafts\-sprecherin rückt die Kandidatin mit den nächstmeisten Stimmen nach. Steht keine Kandidatin mehr zur Verfügung, bleibt das Amt unbesetzt. Fällt die Anzahl der Fach\-schafts\-sprecherin unter zwei, ist eine Fach\-schafts\-versammlung von der noch verbleibenden Fach\-schafts\-sprecherin innerhalb von zwei Wochen in der Vorlesungszeit einzuberufen, um über Neuwahlen zu entscheiden. Ist der Fach\-schafts\-vorstand unbesetzt, regelt die Fach\-schafts\-ordnung das weitere Vorgehen.

Die Fach\-schafts\-ordnung kann vorsehen, dass die jeweiligen studentischen Fakultätsratsmitglieder dem Fachschaftsvorstand angehören.

Die Mitglieder des Fachschaftsvorstands haben das Recht, Anfragen an den  Vorstand und das Studierendenparlament zu stellen. Anfragen sind  schriftlich an die Vorsitzende des betreffenden Organs zu richten.  Anfragen müssen vom Vorstand innerhalb von vier Wochen und vom Studierendenparlament innerhalb von vier Wochen während der Vorlesungszeit in Textform beantwortet werden.

Der Fachschaftsvorstand kann eine Person wählen, die mit beratender Stimme an den Sitzungen des Fakultätsrats teilnimmt.


\Paragraph{title=Fachschaftsversammlung}\label{fachschaft:vv}

Die Fachschaftsversammlung ist das beschließende Organ der Fachschaft.

Jedes Fachschaftsmitglied ist auf der Fachschaftsversammlung stimm- und antragsberechtigt.

Die Fachschaftsversammlung wird mindestens einmal pro Semester und auf Antrag von mindestens 5~\% der Fach\-schaftsmitglieder vom Fachschaftsvorstand einberufen. Bei der Einberufung muss eine Tagesordnung vorgeschlagen sein. Die Fachschaftsordnung hat Regelungen zu Fristen und Bekanntmachungen zutreffen.

Die Fachschaftsversammlung kann Kompetenzen an andere Organe der Fachschaft übertragen. Folgende Kompetenzen sind nicht übertragbar \label{fachschaft:vv:kompetenzen}
  \begin{enumerate}
  \item Beschluss und Änderung der Fachschaftsordnung,
  \item Genehmigung des Haushaltsplans der Fachschaft,
  \item Beschluss einer Neuwahl des Fachschaftsvorstands gemäß \refPar{fachschaft:vv:wahl}, \label{fachschaft:vv:kompetenzen:abwahl}
  \item Einsetzen der Wahlleiterin.\label{fachschaft:vv:wahlleiter}
  \end{enumerate}

Die Fachschaftsversammlung kann mit 10~\% aller Stimmen und Zweidrittel der abgegebenen Stimmen be\-schlie\-ßen, eine Neuwahl des Fach\-schaftsvor\-stands zu veranlassen\label{fachschaft:vv:wahl}.

%
% FACHSCHAFTENKONFERENZ (FSK)
%

\parnumberfalse \section{Fachschaftenkonferenz}\label{fsk}  \parnumbertrue

\Paragraph{title=Aufgaben} \label{fsk:aufgaben}

Die Fachschaftenkonferenz ist ein Organ der Studierendenschaft. Sie vertritt die Interessen der Fachschaften  gegenüber dem Studierendenparlament und dem Vorstand.

Die Fachschaftenkonferenz hat ein Vetorecht gegen Beschlüsse des Studierendenparlaments. Das Veto muss binnen einer Frist von zwei Wochen nach dem Beschluss im Studierendenparlament mit mehr als der Hälfte der satzungsgemäß existierenden Stimmen eingelegt werden. Durch Einlegen des Vetos wird der Beschluss des Studierendenparlaments aufgeschoben. Das Studierendenparlament kann ein Veto mit einer Zweidrittelmehrheit der Stimmberechtigten aufheben.\label{fsk:aufgaben:einspruch}

Legt die Fachschaftenkonferenz ein Veto gegen den Beschluss des Haushalts- oder Wirtschaftsplans ein, so muss sie zugleich einen alternativen Haushalts- oder Wirtschaftsplan beschließen. Über diesen alternativen Haushalts- oder Wirtschaftsplan ist vom Studierendenparlament innerhalb von zwei Wochen zu beschließen. Das Studierendenparlament kann einen neuen Haushalts- oder Wirtschaftsplan beschließen oder mit einer Zweidrittelmehrheit der Stimmberechtigten das Veto gegen den ursprünglichen Haushalts- oder Wirtschaftsplan aufheben. \label{fsk:aufgaben:haushalt}

Abweichend von \refPar{fsk:aufgaben:einspruch} kann das Studierendenparlament ein Veto nicht aufheben, sofern der Beschluss eine Änderung der §§ \refParagraphN{fachschaften:aufgaben} bis \refParagraphN{fsk:organisation} sowie \ref{haushalt:fachschaftsgelder} dieser Satzung beinhaltet.

Die Fachschaftenkonferenz wählt Vertreterinnen in den Finanzausschuss nach \ref{haushalt:finanzausschuss:wahl}.

\Paragraph{title={Zusammensetzung, Stimmverteilung}}

Die Fachschaften entsenden Vertreterinnen in die Fachschaftenkonferenz. Die Vertreterinnen jeder Fachschaft werden vom Fachschaftsvorstand gewählt und müssen von der Fachschaftsversammlung bestätigt werden.\label{fsk:zusammensetzung:vertreter}

Die Innenreferentin soll an den Sitzungen mit beratender Stimme teilnehmen.

Die Verteilung der Stimmen erfolgt unter Beachtung der Anzahl der Studierenden. Die Fachschaften mit
\begin{itemize}
\item bis zu 400 Studierenden haben zwei Stimmen,
\item von 401 bis 800 Studierenden haben drei Stimmen,
\item von 801 bis 1000 Studierenden haben vier Stimmen,
\item von 1001 bis 1300 Studierenden haben fünf Stimmen,
\item von 1301 bis 1600 Studierenden haben sechs Stimmen,
\item von 1601 bis 2000 Studierenden haben sieben Stimmen,
\item von 2001 bis 2500 Studierenden haben acht Stimmen,
\item über 2500 Studierenden haben neun Stimmen.
\end{itemize}

\Paragraph{title=Organisation}\label{fsk:organisation}

Die Fachschaftenkonferenz gibt sich eine Geschäftsordnung.

Die Fachschaftenkonferenz wählt aus ihrer Mitte eine Präsidentin. Die Präsidentin ist für die ordnungsgemäße Einberufung und Durchführung der Sitzungen verantwortlich.

Antragsberechtigt sind
  \begin{enumerate}
  \item die Vertreterinnen der Fachschaften gemäß \ref{fsk:zusammensetzung:vertreter},
  \item der Vorstand der Studierendenschaft,
  \item die Fachschaftsvorstände,
  \item die Mitglieder nach Maßgabe von \ref{studierendenschaft:mitglieder:antraege}.
  \end{enumerate}

Die Fachschaftenkonferenz tagt mindestens einmal pro Vorlesungsmonat.

%
% ARBEITSKREISE UND HOCHSCHULGRUPPEN
%

\parnumberfalse \section{Arbeitskreise und Hochschulgruppen} \parnumbertrue

\Paragraph{title=Arbeitskreise}
\parnumberfalse Zur langfristigen Bearbeitung konkreter Aufgaben oder Teile der Aufgaben nach \ref{studierendenschaft:aufgaben} kann das Studierendenparlament Arbeitskreise der Studierendenschaft einrichten. Diese sind dem Studierendenparlament weisungsgebunden und berichten diesem regelmäßig über ihre Arbeit.\parnumbertrue

\Paragraph{title=Hochschulgruppen}
\parnumberfalse Studentische Gruppen haben die Möglichkeit, sich als Hochschulgruppe der Studierendenschaft beim Vorstand registrieren zu lassen. Voraussetzung sind eine Vereinbarkeit des Zwecks der Hochschulgruppe mit den Aufgaben der Studierendenschaft, dass der Schwerpunkt der Arbeit der Gruppe am KIT liegt und dass die Gruppe selbstlos tätig ist und nicht in erster Linie eigenwirtschaftliche Zwecke verfolgt. Näheres regelt eine gesonderte Satzung.\parnumbertrue

%
% HAUSHALT
%

\parnumberfalse \section{Haushalt} \parnumbertrue

\Paragraph{title=Allgemeines}

Das Studierendenparlament hat die Verfügungsgewalt über das Vermögen der Studierendenschaft.

Das Haushaltsjahr der Studierendenschaft ist das Kalenderjahr.

Das Studierendenparlament erlässt eine Finanzordnung und eine Beitragsordung als Satzungen.

Die Fachschaften haben ein Anrecht auf 20~\% der Einnahmen durch Beiträge der Studierendenschaft. \label{haushalt:fachschaftsgelder}

Der Vorstand legt zum Ende des Geschäftsjahres dem Studierendenparlament und der Fachschaftenkonferenz eine Bilanz vor.

Der Haushalts- oder Wirtschaftsplan und die Bilanz werden veröffentlicht.


\Paragraph{title=Haushalts- oder Wirtschaftsplan} \label{haushalt:haushaltsplan}

Der Vorstand legt dem Studierendenparlament spätestens bis zum 1.~Dezember einen Entwurf des Haushalts- oder Wirtschaftsplans für das folgende Geschäftsjahr vor.

Der Haushalts- oder Wirtschaftsplan muss für jedes Haushaltsjahr ausgeglichen sein.

Außer- und überplanmäßige Ausgaben müssen durch einen Nachtragshaushalt beschlossen werden.

Über das Eröffnen und Schließen von Geschäftsfeldern, sowie grundsätzliche Veränderungen der Wirtschaftsbetriebe, entscheidet das Studierendenparlament. Die Gründung von und die Beteiligung an wirtschaftlichen Unternehmen bedarf darüber hinaus der Zustimmung des Präsidiums des KIT.


\Paragraph{title=Finanzausschuss} \label{haushalt:finanzauschuss}

Der Finanzausschuss unterstützt die Rechnungsprüfung nach §~65~b Absatz~3 Satz~2 LHG. Zusätzlich führt der Finanzausschuss eigene Prüfungen durch. Es erfolgt mindestens eine Prüfung im Semester; über das Ergebnis der Prüfung ist dem Studierendenparlament und der Fachschaftenkonferenz zu berichten. Näheres regelt die Finanzordnung.

Der  Finanzausschuss besteht aus drei durch das Studierendenparlament und  zwei durch die Fachschaftenkonferenz bestimmte Mitglieder. Sie werden nach Maßgabe der Finanzordnung auf ein Jahr gewählt. Die Mitglieder des Finanzausschusses dürfen nicht Mitglied des Vorstands oder erweiterten Vorstands sein. \label{haushalt:finanzausschuss:wahl}


%
% GRUNDSÄTZE
%

\parnumberfalse \section{Grundsätze und Organisatorisches} \parnumbertrue

\Paragraph{title=Wahlen und Abstimmungen}\label{grundsaetze:wahlen}

Wahlen und Abstimmungen der Studierendenschaft finden nach demokratischen Grundsätzen statt. Die Einhaltung demokratischer Regeln ist durch eine geeignete Organisationsweise zu gewährleisten.

Verantwortlich für die Einhaltung demokratischer Regeln bei der Wahl zum Studierendenparlament und zu den Fachschaftsvorständen ist ein vom Studierendenparlament gewählter Wahlausschuss. Er wird bei der Durchführung von den Wahlleiterinnen der Fachschaften nach \ref{fachschaft:vv:kompetenzen} Nummer \ref{fachschaft:vv:wahlleiter} unterstützt. Unmittelbar nach Abschluss der Wahl oder Abstimmung ermittelt der zuständige Ausschuss das Ergebnis und hält es in einer Niederschrift fest, die dem Studierendenparlament und dem Ältestenrat vorgelegt werden muss. Außerdem sorgt er für die unverzügliche Bekanntmachung des Ergebnisses. \label{grundsaetze:wahlen:wahlausschuss}

Bekanntmachungen von Wahlen und Urabstimmungen sind vom Wahlausschuss öffentlich innerhalb des KIT auszuhängen. Mindestens ein Aushang an zentraler Stelle jeder Fakultät sowie der Mensa ist erforderlich. \label{grundsaetze:wahlen:bekanntmachung}

Jedes Mitglied kann eine Wahl oder Abstimmung beim Ältestenrat innerhalb einer Frist von vier Wochen ab der Bekanntmachung des Ergebnisses schriftlich anfechten. Erklärt der Ältestenrat die Wahl oder Abstimmung für ungültig, so ist die Wiederholung unverzüglich auszuschreiben. \label{grundsaetze:wahlen:wahlanfechtung}

Wahlen und Urabstimmungen finden während der vom KIT-Senat beschlossenen Vorlesungszeit an direkt aufeinander folgenden Werktagen statt.

\Paragraph{title=Mehrheiten}
\parnumberfalse In der Regel ist ein Antrag angenommen, wenn ihm mehr anwesende Stimmberechtigte zustimmen, als ihn ablehnen (relative Mehrheit). Folgende Abweichungen von dieser Regel können in Satzungen oder Geschäftsordnungen vorgesehen sein:
\begin{enumerate}
\item Absolute Mehrheit, d.\,h. mehr Ja-Stimmen als die Hälfte der Anzahl der Stimmberechtigten,
\item Zweidrittelmehrheit der abgegebenen Stimmen, d.\,h. mindestens so viele Ja-Stimmen wie zwei Drittel der abgegebenen Stimmen,
\item Zweidrittelmehrheit der Stimmberechtigten, d.\,h. mindestens so viele Ja-Stimmen wie zwei Drittel der Anzahl der Stimmberechtigten.
\end{enumerate}
Als Anzahl der abgegebenen Stimmen gilt die Summe aus Ja-Stimmen, Nein-Stimmen, Enthaltungen und ungültigen Stimmen.\parnumbertrue


\Paragraph{title=In-Kraft-Treten}\label{grundsaetze:inkrafttreten}
\parnumberfalse Diese Satzung tritt am Tage nach ihrer Bekanntmachung in den Amtlichen Bekanntmachungen des KIT in Kraft.\parnumbertrue


\end{contract}

\end{multicols}


\part[Wahl- und Abstimmungsordnung]{Wahl- und Abstimmungsordnung \\ der Verfassten Studierendenschaft \\ des Karlsruher Instituts für Technologie (KIT)}

\begin{multicols}{2}
\begin{contract}
\setcounter{Paragraph}{0}
\setcounter{juratoclevel}{1}

\Paragraph{title={Geltungsbereich}}\label{wahl:geltungsbereich}
Diese Satzung regelt die Wahlen zum Studierendenparlament und den Fachschaftsvorständen, die Urabstimmung, sowie die Wahlen zu weiteren Organen der Studierendenschaft, sofern eine Satzung dies vorsieht.

\Paragraph{title={Wahlsystem}}\label{wahl:wahlsystem}
Das Studierendenparlament wird nach Listen, welche aufgrund gültiger Wahlvorschläge aufgestellt werden, gewählt. 

Bei der Wahl des Studierendenparlament hat jede Wahlberechtigte
\begin{enumerate}
    \item eine Stimme, mit welcher sie eine Liste wählen kann (Listenstimme),
    \item fünf Stimmen mit denen sie Kandidatinnen wählen kann (Personenstimmen); die Stimmen können beliebig kumuliert und panaschiert werden.
\end{enumerate}

Die Fachschaftsvorstände werden in Persönlichkeitswahl von den Fachschaftsmitgliedern gewählt. Die Benennung der Kandidatinnen erfolgt durch die entsprechende Fachschaftsversammlung.

Bei der Wahl der Fachschaftsvorstände hat jede Wählerin so viele Stimmen wie Fachschaftssprecherinnen zu wählen sind. Auf eine Kandidatin dürfen dabei maximal zwei Stimmen kumuliert werden.

\Paragraph{title={Urabstimmung}}\label{wahl:urabstimmung}
Für die Urabstimmung gelten die Regelungen zur Durchführung der Wahlen sinngemäß. 

Die Urabstimmung findet spätestens während der nächsten Wahl zum Studierendenparlament statt. Der Antrag auf Urabstimmung kann den Zeitraum für die Urabstimmung beinhalten. Dabei muss die Einhaltung der Fristen gewährleistet sein.

\Paragraph{title={Wahlberechtigung und Wählbarkeit}}\label{wahl:wahlberechtigung}

Wahlberechtigt sind alle Mitglieder, die im Wählerverzeichnis eingetragen sind.

Wählbar sind alle Mitglieder, welche in einem gültigen Wahlvorschlag aufgeführt sind. Mitglieder des Wahlausschusses sowie des Ältestenrates dürfen in keinen Wahlvorschag aufgenommen werden.

\Paragraph{title={Wahltermin}}\label{wahl:walhtermin}
Die Wahlen sollen jährlich im Sommersemester parallel zur Wahl der studentischen Senatsmitglieder des KIT stattfinden. 

Das Studierendenparlament legt Termin und Dauer der Wahlen fest. Der Wahlzeitraum besteht aus wenigstens drei und höchstens fünf aufeinanderfolgenden Tagen. Die Wahlen dürfen nicht in den ersten drei Vorlesungswochen oder in der letzten Vorlesungswoche des Semesters stattfinden. 

In der Regel finden die Wahlen zum Studierendenparlament und den Fachschaftsvorständen zeitgleich statt.

\Paragraph{title={Wahlausschuss}}\label{wahl:wahlausschuss}
Das Studierenenparlament wählt spätestens 48 Tage vor dem ersten Wahltag bzw. spätestens 27 Tage vor dem ersten Tag der Urabstimmung einen aus vier Personen bestehenden Wahlausschuss. 

Der Wahlausschuss wählt eine Vorsitzende sowie eine Stellvertreterin aus seiner Mitte.

Der Wahlausschuss ist beschlussfähig, wenn mehr als die Hälfte seiner Mitglieder anwesend sind. Beschlüsse werden mit der Mehrheit der anwesenden Mitglieder getroffen. Bei Stimmgleichheit entscheidet die Stimme der Vorsitzenden. Über die Sitzungen ist Protokoll zu führen.

Der Wahlausschuss ist zuständig für
\begin{enumerate}
    \item Bekanntmachung der Wahl bzw. Urabstimmung,
    \item Erstellung des Wählerverzeichnisses,
    \item Annahme, Zulassung und Bekanntmachung der Wahlvorschläge,
    \item Anfertigung der Stimmzettel sowie der weiteren für Wahl und Auszählung erforderlichen Unterlagen,
    \item Beschaffung, Versiegelung und Aufbewahrung der Wahlurnen,
    \item Organisation und Durchführung der Wahl bzw. Urabstimmung sowie der Auszählung,
    \item Feststellung und Bekanntmachung der Wahlergebnisse,
    \item die Einhaltung demokratischer Regeln.
\end{enumerate}

\Paragraph{title={Bekanntmachung der Wahlen}}\label{wahl:bekanntmachung-wahl}
Der Wahlausschuss macht die Wahl spätestens 42 Tage vor dem ersten Wahltag bekannt.

Die Bekanntmachung enthält:
\begin{enumerate}
     \item den Wahlzeitraum sowie die Abstimmungszeiten,
     \item den Hinweis, dass nur wählen darf, wer am Tag des endgültigen Abschlusses des Wählerverzeichnises in diesem eingetragen ist,
     \item Ort, Dauer und Zeit der Einsichtsmöglichkeit in das Wählerverzeichnisses,
     \item die zu wählenden Gremien sowie die Zahl der jeweils zu wählenden Mitglieder und deren Amtszeit,
     \item die Aufforderung, spätestens am 24. Tag vor dem ersten Wahltag Wahlvorschläge beim Wahlausschusses einzureichen, 
     \item den Hinweis, dass Wahlbewerber und Vertreter eines Wahlvorschlags nicht Mitglieder des Wahlausschusses sein können,
     \item den Hinweis auf die Form und den Inhalt der Wahlvorschläge,
     \item die Bestimmungen über die Briefwahl nach \ref{wahl:briefwahl},
     \item den Hinweis darauf, dass keine Bindung an eine bestimmte Wahlurne besteht,
     \item den Hinweis darauf, wo die Wahl- und Abstimmungsordnung einzusehen ist.
\end{enumerate}

Die Bekanntmachung ist nicht vor Ende der Wahl zu entfernen.

\Paragraph{title={Bekanntmachung der Urabstimmung}}\label{wahl:bekanntmachung-ua}
Der Wahlausschuss macht die Urabstimmung spätestens 21 Tage vor dem ersten Wahltag bekannt.

Die Bekanntmachung enthält:
\begin{enumerate}
     \item den Antragstext sowie die Abstimmungsmöglichkeiten,
     \item den Wahlzeitraum sowie die Abstimmungszeiten,
     \item den Hinweis, dass nur wählen darf, wer am Tag des endgültigen Abschlusses des Wählerverzeichnises in diesem eingetragen ist,
     \item Ort, Dauer und Zeit der Einsichtsmöglichkeit in das Wählerverzeichnisses,
     \item die Bestimmungen über die Briefwahl nach \ref{wahl:briefwahl},
     \item den Hinweis darauf, dass keine Bindung an eine bestimmte Wahlurne besteht,
     \item den Hinweis darauf, wo die Wahl- und Abstimmungsordnung einzusehen ist.
\end{enumerate}

Die Bekanntmachung ist nicht vor Ende der Urabstimmung zu entfernen.

\Paragraph{title={Wählerverzeichnis}}\label{wahl:wählernverzeichnis}

Alle Wahlberechtigten sind in ein Wählerverzeichnis in Listenform einzutragen. Die Aufstellung des Wählerverzeichnisses obliegt dem  Wahlausschuss. Es kann im Wahlverfahren auch in elektronischer Form  verwendet werden.

Das Wählerverzeichnis enthält die folgenden Angaben:
\begin{enumerate}
\item laufende Nummer,
\item Familienname,
\item Vorname,
\item Matrikelnummer,
\item Studiengang,
\item Vermerk über die Stimmabgabe,
\item Bemerkungen.
\end{enumerate}

Das Wählerverzeichnis ist sieben Tage nach Bekanntmachung der Wahl bzw. Urabstimmung vorläufig  abzuschließen und für fünf Tage beim Wahlausschuss zur Einsicht durch die Studierenden aufzulegen. Eine Einsichtnahme steht jedem zu, um seine  eigenen Daten auf Richtigkeit und Vollständigkeit zu überprüfen. Zur Überprüfung der Richtigkeit oder Vollständigkeit der Daten von anderen im Wählerverzeichnis eingetragenen Personen haben Wahlberechtigte nur dann ein Recht auf Einsicht in das Wählerverzeichnis, wenn sie Tatsachen glaubhaft machen, aus denen sich eine Unrichtigkeit oder Unvollständigkeit des Wählerverzeichnisses ergeben kann.  \label{wahl:wählerverzeichnis:auflegung}

Das Wählerverzeichnis ist spätestens am 14. Tag nach Bekanntmachung der Wahl bzw. Urabstimmung  unter Berücksichtigung der Entscheidungen nach  \ref{wahl:aenderungwv:aenderungen} vom Wahlausschuss endgültig  abzuschließen. Dabei ist im Wählerverzeichnis
\begin{enumerate}
\item die Zahl der eingetragenen Wahlberechtigten,
\item die Zahl der Anträge auf Berichtigung des Wählerverzeichnisses
\end{enumerate}
vom Wahlausschuss zu beurkunden.

\Paragraph{title={Änderung des Wählerverzeichnisses}}
Das Wählerverzeichnis kann bis zum Ablauf der Einsichtsfrist vom Wahlausschuss berichtigt oder ergänzt werden.

Die  Einsichtsberechtigten gemäß \ref{wahl:wählerverzeichnis:auflegung}  können während der Dauer der Auflegung des Wählerverzeichnisses dessen  Berichtigung oder Ergänzung beantragen, wenn sie diese für unrichtig  oder unvollständig halten. Der Antrag ist schriftlich beim Wahlausschuss  zu stellen. Die erforderlichen Beweise sind vom Antragsteller  beizubringen. Der Wahlausschuss entscheidet spätestens am 14. Tag nach Bekanntmachung der Wahl bzw. Urabstimmung  über die Anträge. Die Entscheidung ist der Antragstellerin und ggf. der  Betroffenen mitzuteilen. \label{wahl:aenderungwv:aenderungen}

Nach  Ablauf der Einsichtsfrist bis zum endgültigen Abschluss des  Wählerverzeichnisses können Eintragungen und Streichungen nur in Vollzug  der Entscheidung gemäß \ref{wahl:aenderungwv:aenderungen} vorgenommen  werden.

Das  Wählerverzeichnis kann bis zum 1. Tag vor dem ersten Wahltag vom  Wahlausschuss bei Vorliegen offensichtlicher Fehler, Unstimmigkeiten  oder Schreibversehen berichtigt oder ergänzt werden.
%soll das "bis zum Tag vor dem ersten Wahltag" heißen?

Änderungen sind als solche kenntlich zu machen und mit Datum und Unterschrift zu versehen.

\Paragraph{title={Wahlvorschläge}}\label{wahl:wahlvorschlaege}

Die Wahlvorschläge sind getrennt für die Wahlen zum Studierendenparlament und den Fachschaftsvorständen spätestens am 24. Tag vor dem ersten Wahltag bis 15:00 Uhr beim Wahlausschuss einzureichen.

Wahlvorschläge für die Wahl zum Studierendenparlament müssen enthalten
\begin{enumerate}
    \item ein Kennwort; Kennwörter dürfen nicht irreführend sein,
    \item eine Liste mit Kandidatinnen; ein Wahlvorschlag darf höchstens so viele Kandidatinnen enthalten, wie Plätze im Studierendenparlament zu besetzen sind, 
    \item eine von mindestens 30 Wahlberechtigten unterzeichnete Unterstützungsliste.
\end{enumerate}

Geben die Kennwörter mehrerer Wahlvorschläge zu Verwechslungen Anlass, so fordert der Wahlausschuss die Vertreterin des später eingereichten Wahlvorschlages unverzüglich auf, sich innerhalb der Mängelbeseitigungsfrist ein anderes Kennwort zu geben.

Ein Wahlvorschlag für den Fachschaftsvorstand wird von der Fachschaftsversammlung erstellt. Er wird von der amtierenden Fachschaftsleiterin unterzeichnet und vertreten. Er beinhaltet:
\begin{enumerate}
    \item eine Liste mit Kandidatinnen,
    \item eine von Sitzungsleitung und Protokollantin unterzeichnete Kopie des Protokolls der Fachschaftsversammlung.
\end{enumerate}

Unterzeichner müssen für die betreffende Wahl wahlberechtigt sein. Sie müssen folgende Angaben machen:
\begin{enumerate}
\item Vor- und Familienname,
\item Matrikelnummer,
\item eigenhändige Unterschrift,
\item bei den ersten beiden Unterstützerinnen: E-Mailadresse und Telefonnummer.
\end{enumerate}
Die erste Unterzeichnerin ist zur Vertretung gegenüber dem Wahlausschuss berechtigt, die zweite Unterzeichnerin vertritt sie.

Eine Wahlberechtigte darf für dieselbe Wahl nicht mehr als einen Wahlvorschlag unterzeichnen. Hat eine Wahlberechtigte dies nicht beachtet, so wird sie von allen eingereichten Wahlvorschlägen gestrichen.

Die Liste der Kandidatinnen muss folgende Angaben zu den Kandidatinnen enthalten:
\begin{enumerate}
\item Laufende Nummer,
\item Vor- und Familienname,
\item Matrikelnummer,
\item Studiengang,
\item E-Mailadresse,
\item eigenhändige Unterschrift.
\end{enumerate}
Die Kandidatinnen bestätigen mit ihrer Unterschrift die Richtigkeit der Daten sowie ihre Zustimmung, auf den Wahlvorschlag aufgenommen zu werden. Eine Kandidatin darf nicht auf mehreren Wahlvorschlägen für dieselbe Wahl aufgenommen werden.

Mitglieder des Wahlausschusses sowie des Ältestenrats dürfen weder auf einem Wahlvorschlag als Kandidatin geführt werden noch einen vertreten.

Die Zurücknahme von Wahlvorschlägen, Unterschriften unter einem Wahlvorschlag oder Zustimmungserklärungen von Bewerbern ist nur bis zum Ablauf der Einreichungsfrist für die Wahlvorschläge zulässig.

Etwaige Mängel am Wahlvorschlag sind der Vertreterin des Wahlvorschlages unverzüglich, spätestens aber am Tag nach Ablauf der Einreichungsfrist mitzuteilen. Danach besteht bis zum Beginn der Wahlausschusssitzung nach \ref{wahl:zulassung:sitzung} die Gelegenheit, die Mängel zu beseitigen. Das Fehlen von erforderlichen Unterschriften gilt nicht als Mangel im oberen Sinne. Diese können nach Ablauf der Einreichungsfrist nicht nachgeholt werden.

\Paragraph{title={Zulassung der Wahlvorschläge}}\label{wahl:zulassung}
Spätestens am 21. Tag vor dem ersten Wahltag beschließt der Wahlausschuss in einer Sitzung über die Zulassung der eingereichten Wahlvorschläge. \label{wahl:zulassung:sitzung}

Zurückzuweisen sind Wahlvorschläge, 
\begin{enumerate}
      \item die nicht fristgerecht eingereicht wurden,
      \item die eine Bedingung enthalten,
      \item die nicht von einer ausreichenden Zahl Wahlberechtigter unterzeichnet wurden, 
      \item welche die Reihenfolge oder die Zuordnung der Personendaten der Kandidatinnen nicht zweifelsfrei erkennen lassen.
\end{enumerate}

In den Wahlvorschlägen sind diejenigen Bewerber zu streichen
\begin{enumerate}
\item die so unvollständig bezeichnet werden, dass Zweifel über ihre Person bestehen,
\item die nicht wählbar sind,
\item deren Zustimmungserklärung nicht ordnungsgemäß vorgelegt wurde oder unter einer Bedingung eingegangen ist,
\item deren Zustimmungserklärung vor Ablauf der Einreichungsfrist der Wahlvorschläge zurückgezogen wurde,
\item die in mehreren Wahlvorschlägen für dieselbe Wahl aufgeführt sind.
\end{enumerate}

Überzählige Kandidatinnen werden in der Reihenfolge von hinten gestrichen.

Die Beschlüsse und deren Begründungen sind in ein Protokoll aufzunehmen.\label{wahl:zulassung:beschluss}

Wird ein Wahlvorschlag zurückgewiesen oder eine Kandidatin gestrichen, so sind die getroffenen Entscheidungen der Vertreterin des Wahlvorschlages sowie der betroffenen Kandidatin unverzüglich mitzuteilen.

Der Wahlausschuss bestimmt unter den Wahlvorschlägen zum Studierendenparlament per Losziehung eine Reihenfolge.

\Paragraph{title={Bekanntmachung der Wahlvorschläge }}\label{wahl:bekanntmachungvorschläge}
Der Wahlausschuss macht die Wahlvorschläge spätestens am 14. Tag vor dem ersten Wahltag bekannt. 
Die Bekanntmachung enthält:
\begin{enumerate}
    \item die zu wählenden Gremien sowie die Zahl der jeweils zu wählenden Mitglieder,
    \item die jeweils zugelassenen Wahlvorschläge in der Reihenfolge nach \ref{wahl:zulassung}, 
    \item den Hinweis, dass nur mit den amtlichen Stimmzetteln des Wahlausschusses gewählt werden darf,
    \item den Hinweis auf die den Wahlberechtigten zur Verfügung stehenden Stimmen sowie ggf. den Hinweis auf die Kumulierbarkeit bzw. Panaschierbarkeit der Personenstimmen,
    \item den Wahlzeitraum sowie die Abstimmungszeiten,
    \item den Hinweis darauf, dass keine Bindung an eine bestimmte Urne besteht,
    \item den Hinweis darauf, dass Studentinnen ihre Wahlberechtigung gemäß \ref{wahl:wahlhandlung:nachweis} nachweisen müssen,
    \item den Hinweis darauf, wo die Wahl- und Abstimmungsordnung einzusehen ist.
\end{enumerate}

Die Bekanntmachung ist nicht vor Ende der Wahl zu entfernen.


\Paragraph{title={Briefwahl}}\label{wahl:briefwahl}
Eine Wahlberechtigte, die zum Zeitpunkt der Wahl verhindert ist, die Abstimmung vor Ort vorzunehmen, erhält auf Antrag in Schriftform beim Wahlausschuss für die Wahl einen Wahlschein und die Briefwahlunterlagen, bestehend aus einem Stimmzettel für jede Wahl, einem Wahlumschlag und einem Wahlbriefumschlag. Die Ausgabe der Wahlscheine und der Briefwahlunterlagen ist im Wählerverzeichnis zu vermerken.

Der Wahlumschlag und der Wahlbriefumschlag müssen als solcher gekennzeichnet sein. Weiter muss der Wahlbriefumschlag die Adresse der Wählerin als Absender und die Adresse des Wahlausschusses als Empfänger ausweisen.\\
Die Briefwählerin trägt die Kosten der Übersendung. Sie ist hierauf hinzuweisen.

Briefwahlunterlagen können frühestens am Tag der Bekanntmachung der Wahl und spätestens am 7. Tag (Eingang beim Wahlausschuss) vor dem ersten Wahltag beantragt werden. \label{wahl:briefwahl:frist}

Bei der Briefwahl kennzeichnet die Wählerin ihren Stimmzettel und steckt ihn in den Wahlumschlag. Sie bestätigt auf dem Wahlschein durch Unterschrift, dass sie den beigefügten Stimmzettel persönlich gekennzeichnet hat und legt den Wahlschein mit dem verschlossenen Wahlumschlag in den Wahlbriefumschlag.

Der Wahlbrief ist an die vorgedruckte Anschrift des Wahlausschusses ausreichend frankiert zu übersenden oder persönlich beim Wahlausschuss abzugeben. Der Wahlausschuss kann der Wahlberechtigten die Möglichkeit geben, bei persönlicher Abholung der Briefwahlunterlagen die Briefwahl an Ort und Stelle auszuüben; in diesem Fall kann die Stimmabgabe auch noch nach Ablauf der Frist nach \ref{wahl:briefwahl:frist} erfolgen. Dabei ist Sorge zu tragen, dass der Stimmzettel unbeobachtet gekennzeichnet und in den Wahlumschlag gelegt werden kann. Der Wahlausschuss nimmt sodann den Wahlbrief entgegen.

Die Stimmabgabe gilt als rechtzeitig erfolgt, wenn der Wahlbrief am letzten Wahltag bis spätestens zum Zeitpunkt des Endes der Abstimmungszeit beim Wahlausschuss eingeht. Auf dem Wahlbriefumschlag ist der Tag des Eingangs, auf den am letzten Wahltag eingehenden Wahlbriefumschlägen zusätzlich die Uhrzeit des Eingangs zu vermerken. Sind eingehende Wahlbriefe unverschlossen, so ist dies auf den Wahlbriefen zu vermerken.

Die eingegangenen Wahlbriefe werden vom Wahlausschuss unter Verschluss ungeöffnet aufbewahrt.

Wahlscheine und Wahlumschläge werden gezählt und die Wahlscheine mit den Eintragungen im Wählerverzeichnis verglichen. Wurde von der Wählerin eine Stimmabgabe an der Urne vorgenommen, so ist ihr Wahlumschlag ungeöffnet zu vernichten.

Die Auszählung der per Briefwahl abgegebenen Stimmen erfolgt entsprechend der Auszählung einer Urne gemäß \ref{wahl:auszaehlung}. Haben weniger als zehn Wählerinnen ihre Stimme per Briefwahl abgegeben, so bestimmt der Wahlausschuss eine Urne, zu der die Stimmzettel aus der Briefwahl hinzugefügt werden.

\Paragraph{title={Stimmzettel }}\label{wahl:stimmzettel}

Der Stimmzettel enthält: 
\begin{enumerate}
    \item die zugelassenen Wahlvorschläge mit ihrem Kennwort und den Kandidatinnen mit vollem Namen und Studienfach; bei den Wahlen zu den Fachschaftsvorständen kann auf die Angabe des Studienfachs verzichtet werden.
    \item einen klar zu erkennenden Platz zum Eintragen der Stimmen durch die Wählerinnen,
    \item den Hinweis auf die zur Verfügung stehenden Stimmen sowie den Hinweis auf die Kumulierbarkeit bzw. Panaschierbarkeit der Personenstimmen,
    \item den Hinweis darauf, dass der Stimmzettel vor dem Einwerfen mit dem Aufdruck nach innen zu falten ist,
    \item den Wahlzeitraum.
\end{enumerate}

\Paragraph{title={Wahlurnen und Urnenbuch}}\label{wahl:urnen}
Der Wahlausschuss legt vor Beginn der Wahl die Anzahl der Wahlurnen fest, versiegelt die Urnen und kennzeichnet sie eindeutig und deutlich sichtbar.

Die Urnen sind so einzurichten, dass die eingeworfenen Stimmzettel nicht vor Ende der Wahl entnommen werden können

Die Urnen sind bis zur Auszählung durch Wahlhelferinnen zu beaufsichtigen oder unter Verschluss zu halten.

Zu jeder Urne ist ein Urnenbuch zu führen. Dieses wird vom Wahlausschuss ausgegeben. In das Urnenbuch ist einzutragen: 
\begin{enumerate}
    \item  der volle Name der für die Urne verantwortlichen Wahlhelferin sowie den Zeitraum der Verantwortlichkeit,
    \item die Unterschrift der verantwortlichen Wahlhelferin als Bestätigung, dass sie die Vorschriften der Wahl- und Abstimmungsordnung kennt und danach handelt,
    \item der volle Name aller weiteren Wahlhelferinnen an der Urne,
    \item Zeitpunkt der Öffnung und Schließung der Urne,
    \item der Aufenthaltsort der Urne,
    \item jede während der Wahl festgestellte Unregelmäßigkeit, welche die Urne betrifft, mit dem Zeitpunkt der Feststellung, dem Namen der Feststellenden und der Beschreibung des Vorgangs,
    \item für jede Wählerin den Namen und die Matrikelnummer.
\end{enumerate}

Die Urnen dürfen das Gelände des KIT nicht verlassen; Ausnahmen regelt der Wahlausschuss. Erstreckt sich eine Wahl oder Abstimmung über mehrere Tage, so sind die Urnen über die Nacht von 20:00 bis 7:00 Uhr unter sicherer Verwahrung zu halten. In dieser Zeit ist keine Wahlhandlung zulässig.

\Paragraph{title={ Wahlhelferinnen }}\label{wahl:wahlhelferinnen}
Der Wahlausschuss bestellt Wahlhelferinnen zur Durchführung der Wahl. Der Wahlausschuss belehrt die Wahlhelferinnen über ihre Pflichten. Der Wahlausschuss kann die Bestellung und Belehrung von Wahlhelferinnen an die Wahlleiterinnen gemäß \ref{grundsaetze:wahlen:wahlausschuss} delegieren.

Die Wahlhelferinnen nehmen ihr Amt unparteiisch und gewissenhaft wahr. Sie enthalten sich während der Ausübung ihres Amtes jeder parteilichen Betätigung. Dazu gehört auch das Tragen von Parteiabzeichen und "~parolen.

\Paragraph{title={ Wahlhandlung }}\label{wahl:wahlhandlung}
Jede Urne wird ständig von einer verantwortlichen Wahlhelferin sowie wenigstens einer weiteren Wahlhelferin betreut. Sind unter den Wahlhelferinnen Kandidatinnen, so müssen diese von unterschiedlichen Wahlvorschlägen stammen. Die Wahlhelferinnen sind für die Einhaltung der Wahl- und Abstimmungsordnung an dieser Wahlurne zuständig und dienen als Ansprechpartnerinnen für die Wählerinnen. Sie sind zur gewissenhaften und unparteiischen Auskunft verpflichtet.

Jede Wahlberechtigte kann ihr Stimmrecht für jede Wahl nur einmal und persönlich wahrnehmen. Wahlberechtigte, die durch körperliche Gebrechen gehindert sind, ihre Stimme allein abzugeben, können sich der Hilfe einer Vertrauensperson bedienen.

Die Wählerin weist sich durch Vorlage des Studierendenausweises oder  eines Immatrikulationsnachweises zusammen mit einem amtlichen Lichtbildausweis aus. \label{wahl:wahlhandlung:nachweis}

Die Wahlhelferinnen nehmen die Daten der Wählerin in das Urnenbuch auf und geben dieser die entsprechenden Stimmzettel.

Die Wahlhelferinnen sorgen für die Möglichkeit einer freien und geheimen Stimmabgabe, beispielsweise durch das Aufstellen von Wahlkabinen.

Beim Einwurf der Stimmzettel markieren die Wahlhelferinnen im Urnenbuch sowie im Wählerverzeichnis die Wahlen, an denen die Wählerin teilgenommen hat.

Im Umkreis von zehn Metern um Wahlurnen ist jede Beeinflussung der Wählerinnen untersagt; es dürfen nur vom Wahlausschuss genehmigte Informationen ausgelegt werden.

\Paragraph{title={ Ende der Wahl, Auszählung }}\label{wahl:auszaehlung}
Die Urnen und Urnenbücher sind nach Ende des Abstimmungszeitraums unverzüglich dem Wahlausschuss zu übergeben.

Die Auszählung soll direkt nach Ende des Abstimmungszeitraums spätestens aber am nächsten Werktag stattfinden. 

Die Auszählung findet öffentlich für Mitglieder der Studierendenschaft statt.

Der Wahlausschuss weist die Auszählhelferinnen ein und überwacht die Auszählung.

Jede Urne wird von mindestens vier Auszählhelferinnen gezählt. Sind unter den Auszählhelferinnen Kandidatinnen, so müssen diese von unterschiedlichen Wahlvorschlägen stammen.

Die Stimmzettel werden der Urne entnommen und gezählt. Ihre Zahl muss mit der Summe der Vermerke im Urnenbuch übereinstimmen. Ergibt sich nach wiederholter Zählung keine Übereinstimmung, so ist dies in der Niederschrift zu vermerken und, soweit möglich, zu erläutern. 

Die Stimmzettel werden auf ihre Gültigkeit überprüft. Ungültige Stimmzettel werden getrennt aufbewahrt und bei der Ermittlung des Abstimmungsergebnisses nicht berücksichtigt. 

Ungültig und bei der Ermittlung der Wahlergebnisse nicht anzurechnen sind Stimmzettel,
\begin{enumerate}
    \item die in Inhalt, Form und Farbe von den bereitgestellten abweichen,
    \item die ganz durchgestrichen oder ganz durchgerissen sind,
    \item die mit Bemerkungen versehen sind, ein auf die Person des Wählenden hinweisendes Merkmal oder einen Vorbehalt enthalten,
    \item aus dem sich der Wille der Wählerin nicht zweifelsfrei ergibt,
    \item deren Stimmverteilung nicht den Vorgaben gemäß \ref{wahl:wahlsystem} entspricht; werden bei der Wahl zum Studierendenparlament entweder zuviele Listen- oder zuviele Personenstimmen abgegeben, so werden nur die jeweils zuviel abgegebenen Listen- oder Personenstimmen für ungültig erklärt; die Gültigkeit der ordnungsgemäß abgegebenen Listen- oder Personenstimmen bleibt davon unberührt.
\end{enumerate}

Für jede Urne wird eine Niederschrift angefertigt. Diese enthält
\begin{enumerate}
    \item für jede Wahl einzeln die Zahl der gültigen und ungültigen Stimmzettel,
    \item für jede Wahl die auf die einzelnen Kandidatinnen entfallenen Stimmen,
    \item für die Wahl zum Studierendenparlament die auf die einzelnen Wahlvorschläge entfallenen Listenstimmen sowie die Enthaltungen bei den Listenstimmen,
    \item die Namen sowie die Unterschriften der Auszählungshelferinnen.
\end{enumerate}

\Paragraph{title={ Verteilung der Sitze und Mandate bei der Wahl zum Studierendenparlament }}\label{wahl:sitzverteilungstupa}
Bei der Wahl zum Studierendenparlament werden die auf die einzelnen Wahlvorschläge entfallenen Sitze nach dem Sainte-Laguë-Verfahren verteilt. Haben mehrere Listen die gleiche 25. Höchstzahl so entscheidet das von der Vorsitzenden des Wahlausschusses zu ziehende Los.

Die bei der Wahl zum Studierendenparlament auf die einzelnen Wahlvorschläge entfallenen Mandate werden den in den Wahlvorschlägen aufgeführten Kandidatinnen in der Reihenfolge der von ihnen erreichten Stimmenzahlen zugeteilt. Bei Stimmgleichheit entscheidet die Reihenfolge der Benennung auf dem Wahlvorschlag.

Kandidatinnen, auf die kein Mandat entfällt, sind in der Reihenfolge der von ihnen erreichten Stimmenzahlen als Ersatzleute ihres Wahlvorschlags festzustellen. 

Entfallen bei der Wahl zum Studierendenparlament auf einen Wahlvorschlag mehr Sitze als Kandidatinnen vorhanden sind, so bleiben die überzähligen Sitze unbesetzt.

\Paragraph{title={ Besetzung der Fachschaftsvorstände }}\label{wahl:fsvorstaende}
Die Anzahl der Fachschaftssprecherinnen wird gemäß \ref{fs:vorstand:anzahl} der Organisationssatzung durch die Fachschaftsordnungen geregelt. Die Kandidatinnen mit den meisten Stimmen werden Fachschaftssprecherinnen und bilden den Fachschaftsvorstand. 

Kandidatinnen, welche nicht Teil des Fachschaftsvorstands werden, werden Ersatzleute in der Reihenfolge der von ihnen erziehlten Stimmenanzahl. 

Bei Stimmgleichheit entscheidet die Reihenfolge der Benennung auf dem Wahlvorschlag.

\Paragraph{title={Feststellung des Wahlergebnisses, Wahlniederschrift }}\label{wahl:wahlniederschrift}
Der Wahlausschuss fertigt eine Wahlniederschrift an. Diese hat insbesondere zu enthalten: 
\begin{enumerate}
    \item die Namen seiner Mitglieder,
    \item den Wahlzeitraum,
    \item Vermerke über gefasste Beschlüsse,
    \item die Beschlüsse und deren Begründungen über die Ablehnung von Wahlvorschlägen oder Kandidatinnen,
    \item die Gesamtzahl der Wahlberechtigten,
    \item für jede Wahl die Zahl der insgesamt abgegebenen gültigen und ungültigen Stimmzettel,
    \item für jede Wahl die Gesamtzahl der gültigen und ungültigen Stimmen, der Enthaltungen sowie die auf die einzelnen Wahlvorschläge bzw. Kandidatinnen entfallenen Stimmen
    \item die Verteilung der Mandate auf die einzelnen Kandidatinnen und die Feststellung der Ersatzleute,
    \item die Unterschriften aller Mitglieder des Wahlausschusses.
\end{enumerate}

Mit der Unterzeichnung der Wahlniederschrift ist das Ergebnis festgestellt.

Im Anschluss an die Feststellung des Wahlergebnisses übergibt der Wahlausschuss dem Ältestenrat alle entstandenen Wahlunterlagen. Dieser hat die Wahlunterlagen zwei Monate lang aufzubewahren und dann zu vernichten. Die Vernichtung der Wahlunterlagen wird ausgesetzt, solange der Ältestenrat noch nicht über eine Anfechtung der Wahl entschieden hat.

\Paragraph{title={Bekanntmachung des Wahlergebnisses }}\label{wahl:ergebnis}
Nach der Feststellung gibt der Wahlausschuss das Wahlergebnis bekannt. Die Bekanntmachung enthält: 
\begin{enumerate}
    \item die Zahl der Wahlberechtigten,
    \item die Zahl der Wählerinnen,
    \item für jede Wahl die Gesamtzahl der gültigen und ungültigen Stimmzettel,
    \item für jede Wahl die Gesamtzahl der gültigen und ungültigen Stimmen sowie die auf die einzelnen Wahlvorschläge bzw. Kandidatinnen entfallenen Stimmen,
    \item für jede Wahl den Prozentsatz der Wahlbeteiligung,
    \item bei der Wahl zum Studierendenparlament: die auf die einzelnen Wahlvorschläge entfallenen Mandate,
    \item bei der Wahl zu den Fachschaftsvorständen: die Zusammensetzung des Fachschaftsvorstands.
\end{enumerate}

Der Wahlausschuss benachrichtigt mit der Bekanntmachung des Wahlergebnisses die gewählten Kandidatinnen. 

\Paragraph{title={Wahlanfechtung}}\label{wahl:wahlanfechtung}
Jedes Mitglied der Studierendenschaft kann die Wahl nach Maßgabe des \ref{grundsaetze:wahlen:wahlanfechtung} der Organisationssatzung anfechten.

\Paragraph{title={Berechnung der Fristen}}\label{wahl:fristen}
Bei der Berechnung der Fristen werden nur Tage gezählt, die in der vom KIT-Senat beschlossenen Vorlesungszeit liegen.

Fällt der letzte Tag einer Frist auf einen vorlesungsfreien Tag, so tritt an dessen Stelle der vorherige Vorlesungstag .


\end{contract}

\end{multicols}


\part{Übergangsbestimmungen}

\begin{multicols}{2}
\begin{contract}
\setcounter{Paragraph}{0}
\setcounter{juratoclevel}{1}

\Paragraph{title={Dauer der ersten Amtsperiode}}\label{amtsperiode}
Die Amtsperiode des ersten Studierendenparlaments beginnt am Tag nach der Feststellung des Wahlergebnisses durch das Präsidium des KIT und endet abweichend von § 16 Absatz 3 der Organisationssatzung am 30. September 2014.

Die Amtsperiode der ersten Fachschaftsvorstände beginnt am Tag nach der Feststellung der Wahlergebnisse durch das Präsidium des KIT und endet abweichend von § 30 Absatz 2 der Organisationssatzung am 30. September 2014. Abweichend von Satz 1 kann die Fachschaftsordnung ein früheres Ende der Amtsperiode vorsehen, sofern sie vorsieht, dass der Fachschaftsvorstand aus den studentischen Fakultätsratsmitgliedern besteht.


\Paragraph{title={Anzahl der Fachschaftssprecherinnen}}\label{fachschaftssprecher}

Bis zum in Kraft treten der Fachschaftsordnung entspricht die Anzahl der Fachschaftssprecherinnen der Anzahl der studentischen Mitglieder des Fakultätsrats gemäß §~9 Absatz~5 Nummer~2 der Gemeinsamen Satzung des KIT vom 21.~März~2011 i.V.m. §~9 Absatz~2 Nummer~2 der Grundordnung der ehemaligen Universität Fridericiana zu Karlsruhe (TH) in der Fassung vom 22. Dezember 2008.


\Paragraph{title={Für die erste Besetzung der Organe erforderliche Wahlen}}\label{wahlen}

Abweichend  von § 5 Absatz 2 der Wahl- und Abstimmungsordnung wird der Termin der ersten Wahlen auf Vorschlag der studentischen Senatsmitglieder von der Präsidentin des KIT festgelegt.

Abweichend von § 6 Absatz 1 der Wahl- und Abstimmungsordnung wird der Wahlausschuss auf Vorschlag der studentischen Senatsmitglieder von der Präsidentin des KIT bestellt.

Abweichend von § 11 Absatz 4 der Wahl- und Abstimmungsordnung müssen Wahlvorschläge für die Wahl zum Fachschaftsvorstand folgendes enthalten
\begin{enumerate}
\item eine Liste mit Kandidatinnen,
\item eine von mindestens 30 Wahlberechtigten unterzeichnete Unterstützungsliste.
\end{enumerate}

Abweichend von § 31 Absatz 4 Satz 2 Nummer 4 der Organisationssatzung werden die Wahlleiterinnen vom Wahlausschuss bestellt.

Zusätzlich zu den Angaben in  § 11 Absatz 7 der Wahl- und Abstimmungsordnung müssen Kandidatinnen ihr Geburtsdatum angeben. Das Geburtsdatum wird ausschließlich zur Bestimmung des lebensältesten Mitglied des jeweiligen Organs verwendet.

Die Aufgaben des Ältestenrates  zur Wahlprüfung nimmt ein auf Vorschlag der studentischen Senatsmitglieder von der Präsidentin des KIT bestellter Wahlprüfungsausschuss wahr.


\Paragraph{title={Konstituierung der Organe}}\label{konstituierung}

Nach der Feststellung der Wahlergebnisse der ersten Wahl beruft das lebensälteste Mitglied des jeweiligen Organs dieses zur konstituierenden Sitzung ein.

Die Studierendenschaft ist konstituiert, wenn sich das letzte der Organe nach § 4 Absatz 1 der Organisationssatzung konstituiert hat. Dabei gilt die Vollversammlung zum Zeitpunkt des In-Kraft-Tretens der Organisationssatzung und der erweiterte Vorstand zum Zeitpunkt der Konstituierung des Vorstandes als konstituiert.

Die Vorsitzende eines Organs teilt dem Vorstand den Zeitpunkt der Konstituierung des Organs mit. Der Vorstand teilt den Zeitpunkt der Konstituierung des letzten Organs nach Absatz 2 dem Präsidium des KIT zur Feststellung und Bekanntmachung mit.


\Paragraph{title={Erlass weiterer Satzungen}}
Das Studierendenparlament soll unverzüglich nach seiner Konstituierung spätestens aber bis zum 31.12.2013 eine Finanzordnung und eine Beitragsordnung erlassen.

Die Fachschaften sollen unverzüglich nach Konstituierung des Fachschaftsvorstands spätestens aber bis zum 31.12.2013 eine Fachschaftsordnung erlassen.

\end{contract}

\end{multicols}

%\fbox{
%\begin{tabular}{ll}
%Impressum & \\
%\small Herausgeber: & UStA KIT \\
%Tel.: & 0721/608-48460 \\
%Web: & www.usta.de \\
%E-Mail: & info$@$usta.de\\
%V.i.S.d.P. & Philipp Rudo\\
%Druck: & Studierenden Service \\
%& Verein (SSV)\\
%Auflage: & 300 Exemplare
%\end{tabular}
%}

\end{document}
