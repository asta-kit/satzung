\stupadate{04.06.2014}
\publishdate{06.06.2014}
\jurchanges{
	\item Amtliche Bekanntmachung 2014 KIT 028 vom 06.06.2014
}


\begin{jurdoc}[Beitragsordnung]{Beitragsordnung\\der Verfassten Studierendenschaft\\des Karlsruher Instituts für Technologie (KIT)}\label{beitragsordnung}

\jurparagraph{Beitragszweck}\label{beitragsordnung:zweck}
Die Verfasste Studierendenschaft des KIT nimmt als eine rechtsfähige Körperschaft des öffentlichen Rechts und Gliedkörperschaft des KIT unbeschadet der Zuständigkeiten des KIT und des Studierendenwerks Karlsruhe Aufgaben nach §~65 Abs.~2 LHG wahr. Um ihre gesetzlichen Aufgaben erfüllen zu können, erhebt die Studierendenschaft gemäß §~65a Abs.~5 Sätze~2 bis 5 LHG unter Berücksichtigung sozialer Belange von ihren Mitgliedern Beiträge nach Maßgabe dieser Beitragsordnung.

\jurparagraph{Beitragspflicht}\label{beitragsordnung:pflicht}
Die Studierendenschaft des KIT erhebt zur Erfüllung ihrer Aufgaben von allen am KIT immatrikulierten Studierenden einschließlich der immatrikulierten Doktorandinnen einen Studierendenschaftsbeitrag. Der Beitragspflicht unterliegen auch die vom Studium beurlaubten Studierenden, nicht jedoch die befristet eingeschriebenen Studierenden nach §~60 Abs.~1 S.~5 LHG.

\jurparagraph{Beitragshöhe}\label{beitragsordnung:hoehe}
Der zu zahlende Studierendenschaftsbeitrag beträgt für jedes Semester 5,99~Euro.

\jurparagraph{Fälligkeit des Beitrags, Einzug durch das KIT}\label{beitragsordnung:faelligkeit}
Der Studierendenschaftsbeitrag ist bei Studierenden zur Neuaufnahme in das KIT mit dem Immatrikulationsantrag beziehungsweise bei bereits eingeschriebenen Studierenden mit der Rückmeldung fällig, ohne dass es eines Gebührenbescheides bedarf. Er ist gemäß §~65a Abs.~5 S.~5 LHG an das KIT zu zahlen, das den Beitrag an die Verfasste Studierendenschaft abführt.

Bei der Einschreibung oder Rückmeldung ist die Zahlung des Beitrages nachzuweisen.

Die Immatrikulation wird gemäß §~60 Abs.~2 Nr.~8 LHG einer Person versagt, die den fälligen Studierendenschaftsbeitrag nicht innerhalb der vom KIT für die Immatrikulation festgesetzten Frist an das KIT gezahlt hat.

Studierende sind vom KIT gemäß §~62 Abs.~2 Nr.~4 LHG von Amts wegen zu exmatrikulieren, wenn sie den Studierendenschaftsbeitrag trotz Mahnung und Androhung der Exmatrikulation nach Ablauf der für die Zahlung gesetzten Frist nicht gezahlt haben.

\jurparagraph{Befreiung, Erlass, Ermäßigung, Stundung, Erstattung}\label{beitragsordnung:befreiung}
Befreiungen vom Studierendenschaftsbeitrag sind nicht vorgesehen. Der Beitrag kann nicht erlassen, ermäßigt oder gestundet werden. Ein Anspruch auf anteilige Rückzahlung des Beitrages im Falle der Exmatrikulation oder Rücknahme der Immatrikulation vor Ablauf des Semesters besteht außer in den Fällen gemäß nachfolgendem \refParL{beitragsordnung:befreiung:exmatrikulation} nicht.

Bei einer Exmatrikulation oder Rücknahme der Immatrikulation binnen eines Monats nach Beginn der Vorlesungszeit am KIT entfällt die Beitragspflicht nach §§~\refParagraphN{beitragsordnung:pflicht},~\refParagraphN{beitragsordnung:hoehe} rückwirkend. Der Studierendenschaftsbeitrag wird auf Antrag für dieses Semester erstattet; ein Anspruch auf einen anteiligen Erlass und eine anteilige Rückerstattung nach Ablauf der Frist in Satz~1 besteht nicht. Der Erstattungsantrag ist binnen einer Frist von einem Monat nach dem Tag der Exmatrikulation an die Verfasste Studierendenschaft zu richten; nach Ablauf dieser Frist besteht ein Anspruch auf Rückerstattung nur noch bei Vorliegen der Voraussetzungen für die Wiedereinsetzung in den vorigen Stand (§~32~LVwVfG). Der Vorstand der Verfassten Studierendenschaft ist berechtigt, im Einvernehmen mit dem KIT die Durchführung des Rückerstattungsverfahrens auf das KIT zu delegieren; Einzelheiten dazu sind in einer Verwaltungsvereinbarung mit dem KIT zu regeln.\label{beitragsordnung:befreiung:exmatrikulation}

\jurparagraph{Inkrafttreten}\label{beitragsordnung:inkrafttreten}
Diese Beitragsordnung tritt am Tag nach ihrer Bekanntmachung in den Amtlichen Bekanntmachungen des KIT in Kraft. Der Studierendenschaftsbeitrag ist erstmals mit der Immatrikulation oder Rückmeldung zum Wintersemester 2014/2015 an das KIT zu bezahlen.

\end{jurdoc}
