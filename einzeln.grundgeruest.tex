%% Dieses LaTeX-File wird nur in die Unterordner gesymlinkt und kompiliert eine
%% einzelne Ordnung

\documentclass[
	a4paper,
	parskip=half,
	numbers=noenddot,
%	titlepage,
%	twoside,  % Benutze diese Optionen für Booklet-Layout
%	BCOR=2mm, % Auskommentiert für bessere Bildschirmdarstellung
	DIV=12,
]{scrartcl}

\synctex=1

%%% rubber: module xelatex
\usepackage{ifxetex}
\ifxetex
	\usepackage{fontspec}% provides font selecting commands
	%\setmainfont[Mapping=tex-text]{Adobe Garamond Pro}
	%\setsansfont[Mapping=tex-text]{Myriad Pro}
\else
	\usepackage[T1]{fontenc}
	\usepackage[utf8]{inputenc}
\fi

\usepackage[ngerman]{babel}

% Wir brauchen scrjura aus den koma-script-Klassen in mindestens Version 0.7
% sonst werden auch einzelne Absätze nummeriert. Falls nicht vorhanden unter
% http://mirrors.ctan.org/install/macros/latex/contrib/koma-script.tds.zip
% runterladen und in ~/texmf/ entpacken (evtl. noch "texhash ~/texmf/" ausführen)
\usepackage[juratotoc,juratocindent=0pt,ref=parlong,ref=nosentence]{scrjura}[2013/11/04]

% Verwende multicol für den Zwei-Spalten-Satz
% Erlaubt Umschaltung innerhalb der Seite und gleicht die Spalten aus
\usepackage{multicol}
% Setze Abstand zwischen Spalten
\setlength{\columnsep}{5mm}

% Für leichteres Programmieren
\usepackage{xifthen}

% Großbuchstaben für die Untergliederungen
\renewcommand{\thesection}{\Alph{section}}

% Setze einen kleinen Abstand \, zwischen Zahl und Buchstabe bei Paragraphen
\renewcommand*{\thecontractSubParagraph}{%
{\theParagraph\texorpdfstring{\,}{}\alph{contractSubParagraph})}}

% Größerer Abstand zwischen Paragraphennummer und -titel im Inhaltsverzeichnis
\renewcommand{\numberline}[1]{\makebox[2.5em][l]{#1}}


% Aus irgendeinem Grund werden chapters nicht als Absatz in einem Paragraphen
% fehlinterpretiert sections aber schon und dementsprechend wird vor ihnen eine
% Absatznummer eingefügt => definiere eigenen Befehl (macht es auch einfacher
% wenn man doch wieder scrbook und chapters will)
\newcommand{\jursection}[1]{\parnumberfalse\section{#1}\parnumbertrue}

% Dieses Kommando setzt einen Paragraphen
% Eigentlich gibt es dazu \Paragraph aus scrjura aber das hat eine komische Argument-Syntax weshalb das nicht ganz so gut mit latexdiff zusammen zu bringen ist und auch irgendwie doof zu benutzen ist. Dieses Kommando reicht eigentlich alle Parameter einfach nur an scrjura weiter aber nimmt den Titel als normales Argument und erlaubt weitere Argumente im optionalen Argument zu übergeben
\newcommand{\jurparagraph}[2][]{%
\ifthenelse{\isempty{#2}}{\Paragraph{#1}}{\Paragraph{title={#2}, #1}}%
}
\newcommand{\jursubparagraph}[2][]{%
\ifthenelse{\isempty{#2}}{\SubParagraph{#1}}{\SubParagraph{title={#2}, #1}}%
}


\usepackage[unicode, pdfusetitle, pdfborder={0 0 0}, bookmarksnumbered]{hyperref}


% Befehle für Metainformationen
\makeatletter
\newcommand*{\@stupadate}{Noch nicht beschlossen}
\newcommand*{\@publishdate}{Noch nicht veröffentlicht}
\newcommand*{\@jurchanges}{Noch nicht veröffentlicht}
\newcommand*{\stupadate}[1]{\renewcommand*{\@stupadate}{#1}}
\newcommand*{\publishdate}[1]{\renewcommand*{\@publishdate}{#1}}
\newcommand*{\jurchanges}[1]{\renewcommand*{\@jurchanges}{Eingearbeitete Änderungen:\begin{itemize}#1\end{itemize}}}

% Setze Titel des Dokuments als Kopfzeile
\usepackage{scrlayer-scrpage}
\pagestyle{scrheadings}
\def\@jurshorttitle{}
\lohead{\@jurshorttitle}
\rohead{Beschlossen: \@stupadate}


% Diese Umgebung umschließt das eigentliche Dokument. Argument ist der Titel, optional noch ein Kurztitel
\newenvironment{jurdoc}[2][]{% begin{jurdoc}
\ifthenelse{\isempty{#1}}{\def\@jurshorttitle{#2}}{\def\@jurshorttitle{#1}}

\title{#2}
\author{}
\date{
	\begin{tabular}{ll}
	Beschluss: & \@stupadate \\
	Veröffentlichung: & \@publishdate \\
	\end{tabular}
}


\maketitle

\@jurchanges

Im Folgenden wird aus Gründen der besseren Lesbarkeit ausschließlich die weibliche Form verwendet. Dabei ist jede andere Form impliziert. Die Geschlechtsdefinition obliegt jeder Person selbst.

\begin{multicols}{2}
\tableofcontents
\end{multicols}

\bigskip

\begin{multicols}{2}%
\begin{contract}%
\setcounter{Paragraph}{0}%
%
}{% end{jurdoc}
\end{contract}%
\end{multicols}%
}% Ende von jurdoc

\makeatother

\begin{document}


\stupadate{18.06.2014} % Hier trägt man das Datum des Beschlusses im StuPa ein
\publishdate{18.06.2014} % Hier trägt man das Datum der Bekanntmachung/des Inkrafttreten ein
\jurchanges{
	% Hier trägt man die Änderungen ein die eingearbeitet wurden
	%\item Amtliche Bekanntmachung 2013 KIT 004 vom 04.02.2013
	%\item Amtliche Bekanntmachung 2014 KIT 021 vom 02.05.2014
	\item Beschlossene Fassung aus der StuPa-Sitzung vom 17.06.2014, ÄRa-Sitzung vom 04.06.2014
}


% Hier beginnt das eigentliche Dokument.
% Argument ist der Titel des Dokuments. In eckigen Klammern falls benötigt ein
% Kurztitel der dann statt des langen in Kopfzeilen oder im Inhaltsverzeichnis
% angezeigt wird
\begin{jurdoc}[Geschäftsordnung des Ältestenrates]{Geschäftsordnung des Ältestenrates\\der Verfassten Studierendenschaft\\des Karlsruher Instituts für Technologie (KIT)}


\jurparagraph{Einberufung}\label{aera-go:einberufung}

Der Ältestenrat ist von einem Mitglied des Ältestenrates einzuberufen. Einzuladen sind alle Mitglieder des Ältestenrates.

Zur Sitzung ist mit einer Frist von 3~Tagen öffentlich einzuladen. Wenn alle Mitglieder des Ältestenrates zustimmen kann auch ohne Frist eingeladen werden. Die Abweichung von der Frist muss im Protokoll begründet werden.


\jurparagraph{Tagesordnung}\label{aera-go:tagesordnung}

In der Tagesordnung sind alle Anträge zu berücksichtigen, die vor Beginn der Sitzung eingereicht wurden.

Zusätzliche Tagesordnungspunkte können während der Sitzung aufgenommen werden. Auf Antrag eines nicht anwesenden Mitglieds des Ältestenrates werden solche Tagesordnungspunkte auf der folgenden Sitzung neu behandelt.


\jurparagraph{Öffentlichkeit}\label{aera-go:oeffentlichkeit}

Der Ältestenrat tagt in der Regel öffentlich. Alle Anwesenden haben Rederecht.

Die Öffentlichkeit oder Teile der Öffentlichkeit können für einzelne Tagesordnungspunkte ausgeschlossen werden, sofern personenbezogene Sachverhalte oder solche Sachverhalte, die aufgrund von Gesetzen oder anderer Rechtsnormen als vertraulich einzustufen sind, behandelt werden.


\jurparagraph{Beschlussfähigkeit}\label{aera-go:beschlussfaehigkeit}

Der Ältestenrat ist beschlussfähig, wenn mehr als die Hälfte der Mitglieder anwesend sind und ordnungsgemäß eingeladen wurde.


\jurparagraph{Persönliche Erklärungen}\label{aera-go:pers_erklaerungen}

Persönliche Erklärungen können von Mitgliedern des Ältestensrates in Textform abgegeben werden.

Persönliche Erklärungen sind im Protokoll am Ende des jeweiligen Tagesordnungspunktes anzuhängen, sofern in der persönlichen Erklärung weder Personen namentlich genannt werden noch diskriminierende Inhalte oder Beleidigungen enthalten sind.


\jurparagraph{Protokoll}\label{aera-go:protokoll}

Von jeder Sitzung des Ältestenrates ist ein Beschlussprotokoll anzufertigen, das mindestens Folgendes enthält:
\begin{enumerate}
	\item Datum, Beginn und Ende der Sitzung
	\item Anwesenheitsliste
	\item Protokollantin der Sitzung
	\item die Tagesordnung
	\item alle Anträge
	\item alle Beschlüsse
	\item persönliche Erklärungen
\end{enumerate}

Das Protokoll ist bis spätestens zwei Wochen nach der Sitzung fertigzustellen und den Mitgliedern des Ältestenrates in Textform zur Genehmigung vorzulegen. Sofern nicht binnen drei Tagen Widerspruch eingelegt wird, gilt das Protokoll als genehmigt.

Das Protokoll ist spätestens eine Woche nach der Genehmigung in geeigneter Weise zu veröffentlichen.


\jurparagraph{Inkrafttreten}\label{aera-go:inkrafttreten}

Diese Geschäftsordnung tritt am Tage nach dem Beschluss im Studierendenparlament in Kraft.



% Hier ist das Dokument zu Ende
\end{jurdoc}



\end{document}
