\stupadate{22.01.2013}
\publishdate{04.02.2013}
\jurchanges{
	\item Amtliche Bekanntmachung 2013 KIT 004 vom 04.02.2013
}

\jurtitle[Organisationssatzung]{Organisationssatzung\\der Verfassten Studierendenschaft\\des Karlsruher Instituts für Technologie (KIT)}

\begin{jurdoc}
\setcounter{juratoclevel}{2}

%
% STUDIERENDENSCHAFT
%
\jursection{Studierendenschaft}
\jurparagraph{Studierendenschaft}
Die immatrikulierten Studierenden einschließlich der immatrikulierten Doktorandinnen des Karlsruher Instituts für Technologie (KIT) (Mitglieder) bilden gemäß § 65 Absatz 1 LHG die Studierendenschaft. Sie ist eine rechtsfähige Körperschaft des öffentlichen Rechts und als solche eine Gliedkörperschaft des KIT. Die Studierendenschaft arbeitet auf demokratischer Grundlage und wahrt nach den verfassungsrechtlichen Grundsätzen die weltanschauliche, religiöse und parteipolitische Neutralität.


\jurparagraph{Aufgaben}\label{studierendenschaft:aufgaben}

Die Studierendenschaft verwaltet ihre Angelegenheiten im Rahmen der gesetzlichen Bestimmungen selbst. Sie hat unbeschadet der Zuständigkeit der Hochschule und des Studentenwerks die folgenden Aufgaben
\begin{enumerate}
\item die Wahrnehmung der hochschulpolitischen, fachlichen und fachübergreifenden sowie der sozialen, wirtschaftlichen und kulturellen Belange der Studierenden,
\item die Mitwirkung an den Aufgaben des KIT nach §§~2~bis~7~LHG i.~V.~m. §~20~KITG,
\item die Förderung der politischen Bildung und des staatsbürgerlichen Verantwortungsbewusstseins der Studierenden,
\item die Förderung der Gleichstellung und den Abbau von Benachteiligungen innerhalb der Studierendenschaft,
\item die Förderung der sportlichen und musischen Aktivitäten der Studierenden,
\item die Pflege und der Ausbau der überregionalen und internationalen Studierendenbeziehungen.
\end{enumerate}

\jurparagraph{Rechte und Pflichten der Mitglieder}\label{studierendenschaft:mitglieder}

Jedes Mitglied hat das aktive Wahlrecht.

Soweit diese Satzung keine Einschränkungen vorsieht, hat jedes Mitglied das passive Wahlrecht.

Jeweils 25 Mitglieder haben das Recht, Anfragen an die Organe nach \ref{organe:organe} Nummer \ref{organe:organe:stupa}, \ref{organe:organe:vorstand}, \ref{organe:organe:aera} und \ref{organe:organe:fsk} zu stellen. Anfragen sind schriftlich an die Vorsitzende des betreffenden Organs zu richten.
Anfragen müssen vom Vorstand innerhalb von vier Wochen und von den anderen Organen innerhalb von vier Wochen während der Vorlesungszeit in Textform beantwortet werden.\label{studierendenschaft:mitglieder:anfragen}

Jeweils 25 Mitglieder haben ein Antragsrecht an die Organe nach \ref{organe:organe} Nummer \ref{organe:organe:stupa}, \ref{organe:organe:vorstand}, \ref{organe:organe:aera} und \ref{organe:organe:fsk}. Anträge sind schriftlich an die Vorsitzende des betreffenden Organs zu richten.\label{studierendenschaft:mitglieder:antraege}

Jedes Mitglied hat das Recht der Beschwerde gegen Maßnahmen und Beschlüsse der Organe der Studierendenschaft, insbesondere mit der Behauptung, es liege ein Verstoß gegen die Organisationssatzung vor. Beschwerden sind schriftlich an den Ältestenrat zu richten. \label{studierendenschaft:mitglieder:beschwerden}

\jurparagraph{Organe der Studierendenschaft}

Die Organe der Studierendenschaft auf zentraler Ebene sind \label{organe:organe}
\begin{enumerate}
\item die Vollversammlung,
\item das Studierendenparlament, \label{organe:organe:stupa}
\item der Vorstand, \label{organe:organe:vorstand}
\item der erweiterte Vorstand,
\item der Ältestenrat, \label{organe:organe:aera}
\item die Fachschaftenkonferenz. \label{organe:organe:fsk}
\end{enumerate}

Die Organe der Studierendenschaft tagen grundsätzlich öffentlich. Die Geschäftsordnung des jeweiligen Organs kann in begründeten Fällen -- insbesondere in Personalangelegenheiten und aus datenschutzrechtlichen Gründen -- Ausnahmen hiervon vorsehen.

Über die Sitzungen der Organe sind Protokolle anzufertigen; diese müssen veröffentlicht werden. Näheres regelt die Geschäftsordnung des jeweiligen Organs.

%
% URABSTIMMUNG
%

\jursection{Urabstimmung}

\jurparagraph{Aufgaben}
Die Urabstimmung entscheidet über grundlegende Fragen der Studierendenschaft. Sie kann über die Auflösung des Studierendenparlaments sowie über Änderungen der Organisationssatzung entscheiden.

\jurparagraph{Stimmrecht}
Jedes Mitglied ist bei der Urabstimmung stimmberechtigt.

\jurparagraph{Zustandekommen}\label{urabstimmung:zustandekommen}
Die Urabstimmung findet statt
\begin{enumerate}
\item auf Beschluss des Studierendenparlaments,
\item auf Beschluss der Fachschaftenkonferenz mit mehr als der Hälfte der satzungsgemäß existierenden Stimmen, \label{urabstimmung:zustandekommen:fsk}
\item auf Antrag der Mitglieder; zu diesem Zweck muss ein entsprechender Antrag von mindestens 5~\% der Mitglieder unterzeichnet sein; der Antrag ist schriftlich beim Ältestenrat einzureichen; dieser beantragt nach Überprüfung der Voraussetzungen unverzüglich eine Sitzung des Studierendenparlaments zur Wahl des Wahlausschusses. \label{urabstimmung:zustandekommen:mitglieder}
\end{enumerate}

\jurparagraph{Organisation und Ablauf}\label{urabstimmung:orga}

Findet gemäß \ref{urabstimmung:zustandekommen} eine Urabstimmung statt, so wählt das Studierendenparlament unverzüglich einen Wahlausschuss für die Durchführung der Urabstimmung; dazu ist der Beschluss nach \ref{urabstimmung:zustandekommen} Nummer \ref{urabstimmung:zustandekommen:fsk} bzw. die Erfüllung  der Bedingungen nach \ref{urabstimmung:zustandekommen}  Nummer \ref{urabstimmung:zustandekommen:mitglieder} dem Präsidium des Studierendenparlaments unverzüglich mitzuteilen.

Es gelten die Vorschriften des \ref{grundsaetze:wahlen}. Darüber hinaus regelt die Wahl- und Abstimmungsordnung weitere Einzelheiten.


\jurparagraph{Beschlüsse}

Beschlüsse der Urabstimmung sind gültig und bindend für die Organe der Studierendenschaft, wenn mindestens ein Sechstel aller Mitglieder sowie die Mehrheit der an der Abstimmung teilnehmenden Mitglieder zugestimmt haben.

Die Urabstimmung entscheidet bei Änderungen der Organisationssatzung mit Zweidrittelmehrheit der abgegebenen Stimmen.

Beschlüsse der Urabstimmung heben widersprechende Beschlüsse der Vollversammlung und des Studierendenparlaments auf.

%
% VOLLVERSAMMLUNG
%

\jursection{Vollversammlung}

\jurparagraph{Aufgaben}

Die Vollversammlung ist ein beschließendes Organ der Studierendenschaft und dient der Information der Mitglieder.

Die Vollversammlung kann nicht über Änderungen der Organisationssatzung sowie Erlass und Änderung weiterer Satzungen, der Finanzordnung, der Beitragsordnung und des Haushalts- oder Wirtschaftsplans beschließen.

\jurparagraph{Stimm- und Antragsrecht}

Jedes Mitglied ist auf der Vollversammlung stimm- und antragsberechtigt.

\jurparagraph{Zustandekommen} \label{vv:zustandekommen}

Eine Vollversammlung findet statt
\begin{enumerate}
\item auf Beschluss des Studierendenparlaments,
\item auf Beschluss der Fachschaftenkonferenz mit mehr als der Hälfte der satzungsgemäß existierenden Stimmen,
\item \label{vv:zustandekommen:mitglieder}auf Antrag der Mitglieder; zu diesem Zweck muss ein entsprechender Antrag von mindestens 2~\% der Mitglieder unterzeichnet sein; der Antrag ist schriftlich beim Ältestenrat einzureichen.
\end{enumerate}


\jurparagraph{Organisation und Ablauf}\label{vv:organisation}

Die Organisation der Vollversammlung obliegt dem Ältestenrat; er kann den Vorstand damit beauftragen.

Die Vollversammlung findet spätestens 30~Tage nach dem Beschluss des Studierendenparlaments oder der Fachschaftenkonferenz bzw. dem Eingang des Antrags der Mitglieder statt, sofern im Beschluss oder Antrag kein Zeitpunkt genannt ist oder der genannte Zeitpunkt die rechtzeitige Einladung nicht zulässt.

Die Einladung zur Vollversammlung erfolgt durch eine Bekanntmachung entsprechend \ref{grundsaetze:wahlen:bekanntmachung} mit einer Frist von einer Woche. Die Bekanntmachung enthält einen Vorschlag für die Tagesordnung, der alle auf Einberufungsanträgen gewünschten Tagesordnungspunkte enthalten muss.

Vollversammlungen sind öffentlich. Die anwesenden Mitglieder haben Rederecht. Nichtmitglieder können auf Antrag von der Vollversammlung ausgeschlossen werden.

Zu Beginn der Vollversammlung wird ein Präsidium gewählt. Der Ältestenrat macht hierzu einen Vorschlag. Dem Präsidium darf kein Mitglied des Ältestenrates angehören. Das Präsidium besteht aus einer Präsidentin und drei Stellvertreterinnen. Das Präsidium ist für die ordnungsgemäße Durchführung der Vollversammlung verantwortlich.

Das Protokoll der Vollversammlung ist binnen einer Woche fertigzustellen und dem Studierendenparlament vorzulegen.

Das Studierendenparlament kann eine Geschäftsordnung für die Vollversammlung beschließen. Ist eine solche nicht vorhanden, so findet die Ge\-schäfts\-ord\-nung des Studierendenparlaments sinngemäß Anwendung. Die Vollversammlung kann Abweichungen von der Geschäftsordnung beschließen.



\jurparagraph{Beschlüsse}

Beschlüsse der Vollversammlung sind gültig und wirksam, wenn mindestens 5~\% aller Mitglieder sowie die Mehrheit der an der Abstimmung teilnehmenden Mitglieder zugestimmt haben. Erreicht ein Beschluss dieses Quorum nicht, so behandelt das Studierendenparlament diesen auf seiner nächsten Sitzung.

Beschlüsse der Vollversammlung heben widersprechende Beschlüsse des Studierendenparlaments auf.

Beschlüsse der Vollversammlung sind ausgesetzt, sobald eine Urabstimmung dazu beantragt ist.


%
% STUDIERENDENPARLAMENT
%

\jursection{Studierendenparlament}

\jurparagraph{Aufgaben}
Das Studierendenparlament ist das beschließende Organ der Studierendenschaft; es ist das legislative Organ gemäß §~65~a Absatz~3 Satz~2~LHG.

Das Studierendenparlament ist insbesondere zuständig für
\begin{enumerate}
\item die Wahl und Abberufung der Vorstandsmitglieder,
\item die Wahl des Ältestenrats,
\item Änderungen der Organisationssatzung,
\item den Beschluss sonstiger Satzungen,
\item den Beschluss über den Haushalt der Studierendenschaft,
\item die Entscheidung über die Führung eines Wirtschaftsplans (§~110~LHO) anstelle eines Haushaltsplans (§~106~LHO),
\item den Beschluss über alle sonstigen Maßnahmen, die die Studierendenschaft langfristig finanziell belasten,
\item den Zusammenschluss mit studentischen Vertretungen anderer Hochschulen,
\item die Wahl des Wahlausschusses,
\item die Wahl von Vertreterinnen in den Finanzausschuss nach \ref{haushalt:finanzausschuss:wahl},
\item die Wahl von studentischen Mitgliedern in Gremien auf zentraler Ebene des KIT, soweit hierzu keine direkten Wahlen stattfinden.
\end{enumerate}


\jurparagraph{Zusammensetzung, Wahl}\label{stupa:zusammensetzung}

Das Studierendenparlament besteht aus 25 Abgeordneten, die von den Mitgliedern der Studierendenschaft nach den Grundsätzen der Verhältniswahl allgemein, gleich, frei, geheim und unmittelbar gewählt werden. Es gelten die Vorschriften des \ref{grundsaetze:wahlen}. Darüber hinaus regelt die Wahl- und Abstimmungsordnung weitere Einzelheiten.

Eine Abgeordnete scheidet aus \label{stupa:zusammensetzung:ausscheiden}
  \begin{enumerate}
  \item am Ende der Amtsperiode,
  \item durch Exmatrikulation,
  \item durch eigenen Verzicht; dieser ist dem Präsidium des Studierendenparlaments in Textform mitzuteilen,
  \item bei Auflösung des Studierendenparlaments,
  \item durch automatischen Ausschluss bei dreimaligem unentschuldigtem Fehlen bzw. bei insgesamt fünfmaliger Abwesenheit von den Sitzungen des Studierendenparlaments; die Feststellung erfolgt durch das Präsidium des Studierendenparlaments; näheres regelt die Geschäftsordnung; liegen triftige Gründe für das Fehlen vor, kann der Ältestenrat innerhalb von 14 Tagen die Wiederanerkennung des Sitzes verfügen; nachgerückte Abgeordnete verlieren in diesem Falle wieder ihren Sitz. \label{stupa:zusammensetzung:ausscheiden:wiederanerkennung}
\end{enumerate}
Bei Ausscheiden einer Abgeordneten rückt die Nächste auf der Liste nach. Ist die Liste erschöpft, so bleibt der Sitz unbesetzt.

Die Amtsperiode des Studierendenparlaments beginnt in der Regel am 1. Oktober und endet am darauffolgenden 30. September.


\jurparagraph{Organisation und Ablauf}

Das Studierendenparlament gibt sich eine Geschäftsordnung.

Das Studierendenparlament wählt sich in jeder Amtsperiode aus seiner Mitte ein Präsidium. Das Präsidium besteht aus einer Präsidentin und zwei Stellvertreterinnen. Das Präsidium ist für die ordnungsgemäße Einberufung und Durchführung der Sitzungen verantwortlich. Seine Mitglieder haben in der Studierendenschaft uneingeschränktes Informationsrecht.

Antragsberechtigt in Sitzungen des Studierendenparlaments sind
  \begin{enumerate}
  \item die Abgeordneten,
  \item die Mitglieder des Vorstandes der Studierendenschaft,
  \item der Ältestenrat,
  \item die Fachschaftsvorstände,
  \item die Fachschaftenkonferenz,
  \item die Präsidentin der Fachschaftenkonferenz,
  \item die Mitglieder nach Maßgabe von \ref{studierendenschaft:mitglieder:antraege}.
  \end{enumerate}

Das Studierendenparlament tagt mindestens einmal pro Vorlesungsmonat. Darüber hinaus muss es auf Antrag des Vorstands, des Ältestenrats oder eines Viertels der Abgeordneten einberufen werden.

Das Studierendenparlament wird von der Präsidentin in Textform einberufen. Mit der Einberufung ist die vorgeschlagene Tagesordnung bekanntzumachen.

Die Abgeordneten sind verpflichtet, an jeder Sitzung persönlich teilzunehmen. Das Stimmrecht kann nicht delegiert werden. Entschuldigungen sind beim Präsidium vor der Sitzung in Textform einzureichen.

Die Abgeordneten haben das Recht, Anfragen an den Vorstand zu stellen. Anfragen sind schriftlich an die zuständige Referentin zu richten und müssen innerhalb von vier Wochen in Textform beantwortet werden.

Die Abgeordneten haben das Recht, Einsicht in die Unterlagen des Vorstands zu verlangen. Der Vorstand hat das Verlangen binnen zwei Wochen zu erfüllen, indem er die Unterlagen in seinen Räumen zur Einsicht vorlegt. Enthalten die Unterlagen personenbezogene Daten, so Bedarf die Einsicht der Zustimmung der betroffenen Personen.


\jurparagraph{Beschlüsse}\label{stupa:beschluesse}

Das Studierendenparlament ist beschlussfähig, wenn mehr als die Hälfte der Mitglieder des Studierendenparlaments anwesend sind. Wird zu Beginn oder während der Sitzung festgestellt, dass das Studierendenparlament nicht beschlussfähig ist, so wird die Sitzung vertagt. Das Studierendenparlament ist auf der nächsten Sitzung in Bezug auf die vertagten Punkte, unbeschadet \refPar{stupa:beschluesse:zweidrittel}, beschlussfähig.

Für folgende Beschlüsse ist eine Zweidrittelmehrheit der Stimmberechtigten des Studierendenparlaments erforderlich \label{stupa:beschluesse:zweidrittel}
\begin{enumerate}
  \item Selbstauflösung des Studierendenparlaments,
  \item Änderung der Organisationssatzung oder der Erlass bzw. Änderung weiterer Satzungen sowie der Ge\-schäfts\-ord\-nungen von Studierendenparlament und Vollversammlung,
  \item Änderung des Haushalts- oder Wirtschaftsplans,
  \item Aufhebung eines Vetos der Fachschaftenkonferenz nach \ref{fsk:aufgaben:einspruch}.
\end{enumerate}


%
% VORSTAND
%

\jursection{Vorstand}

\jurparagraph{Aufgaben}

Der Vorstand ist das ausführende Organ der Studierendenschaft; er ist das exekutive Kollegialorgan gemäß §~65~a Absatz~3 Satz~3~LHG.

Der Vorstand führt die laufenden Geschäfte in eigener Verantwortung im Rahmen der Beschlüsse von Studierendenparlament, Vollversammlung und Urabstimmung. Er ist dem Studierendenparlament rechenschaftspflichtig.

Der Vorstand wählt aus seiner Mitte eine Person, die mit beratender Stimme an den Sitzungen des Senats teilnimmt.

Der Vorstand vertritt die Studierendenschaft in der landesweiten Vertretung der Studierendenschaften nach §~65~a Absatz~8~LHG.

Der Vorstand kann sich eine Geschäftsordnung geben.


\jurparagraph{Zusammensetzung, Wahl} \label{vorstand:zusammensetzung}

Der Vorstand der Studierendenschaft besteht in der Regel aus folgenden Referaten \label{vorstand:zusammensetzung:regel}
\begin{enumerate}
\item Vorsitz,
\item Finanzen,
\item Inneres,
\item Soziales I,
\item Soziales II,
\item Äußeres,
\item Ökologie,
\item Presse und Öffentlichkeitsarbeit,
\item Kultur,
\item Chancengleichheit,
\item Ausländerinnen.
\end{enumerate}
Veränderungen dieser Struktur können vom Studierendenparlament mit absoluter Mehrheit beschlossen werden; die Referate Vorsitz, Finanzen, Chancengleichheit und Ausländerinnen bleiben hiervon unberührt. Die Anzahl der Referate darf zwölf nicht übersteigen.

Das Studierendenparlament besetzt zu Beginn seiner Amtszeit die Referate durch Wahl in getrennten Wahlgängen mit je einem Mitglied der Studierendenschaft. Einem Antrag auf geheime Wahl muss stattgegeben werden\label{vorstand:zusammensetzung:wahl}.

Der Vorstand ist im Amt, wenn Vorsitz und Finanzreferat besetzt sind.

Die Vorsitzende vertritt die Studierendenschaft. Ist die Vorsitzende  verhindert wird sie durch die Finanzreferentin vertreten, es sei denn der Vorstand hat vorher ausdrücklich eine andere Referentin bestimmt.

Die Vorstandsmitglieder scheiden aus
  \begin{enumerate}
  \item mit der Wahl eines neuen Vorstands gemäß \refPar{vorstand:zusammensetzung:wahl},
  \item durch Exmatrikulation,
  \item durch eigenen Verzicht,
  \item durch konstruktives Misstrauensvotum des Studierendenparlaments.
  \end{enumerate}
    Ist ein Referat nach \ref{vorstand:zusammensetzung:regel} nicht besetzt, führt das Studierendenparlament eine Nachwahl für den Rest der Amtszeit durch.

Ist das Chancengleichheitsreferat durch einen Mann besetzt, muss eine Frau zur Unterstützung gemäß \ref{erweitertervorstand:wahl} in den erweiterten Vorstand gewählt werden; ist es durch eine Frau besetzt, muss entsprechend ein Mann gewählt werden.


%
% ERWEITERTER VORSTAND
%

\jursection{Erweiterter Vorstand}

\jurparagraph{Aufgaben}\label{erweitertervorstand:aufgaben}
Der erweiterte Vorstand unterstützt den Vorstand bei seiner Arbeit. Er ist diesem rechenschaftspflichtig.

\jurparagraph{Wahl} \label{erweitertervorstand:wahl}

Die Mitglieder des erweiterten Vorstands werden vom Vorstand gewählt. Diese müssen vom Studierendenparlament einzeln bestätigt werden, einem Antrag auf geheime Abstimmung muss stattgegeben werden.

Die Mitglieder des erweiterten Vorstands scheiden aus
\begin{enumerate}
  \item mit der Wahl eines neuen Vorstands gemäß \ref{vorstand:zusammensetzung:wahl},
  \item durch Exmatrikulation,
  \item durch eigenen Verzicht,
  \item durch Beschluss des Vorstandes mit absoluter Mehrheit,
  \item durch Beschluss des Studierendenparlaments mit absoluter Mehrheit.
\end{enumerate}


%
% ÄLTESTENRAT
%

\jursection{Ältestenrat}

\jurparagraph{Aufgaben}\label{aera:aufgaben}

Der Ältestenrat ist die Schlichtungskommission gemäß §~65~a Absatz~9~LHG. Darüber hinaus hat er folgende Aufgaben: \label{aera:aufgaben:allgemein}
  \begin{enumerate}
  \item Aufhebung satzungswidriger Beschlüsse gemäß \ref{studierendenschaft:mitglieder:beschwerden},
  \item \label{aera:aufgaben:vv} Organisation einer Vollversammlung gemäß \ref{vv:organisation},
  \item \label{aera:aufgaben:ua} Entgegennahme und Prüfung eines Antrags auf Urabstimmung  gemäß \ref{urabstimmung:zustandekommen} Nummer \ref{urabstimmung:zustandekommen:mitglieder} oder Vollversammlung gemäß \ref{vv:zustandekommen} Nummer \ref{vv:zustandekommen:mitglieder},
  \item Entscheidung über die Anfechtung einer Wahl oder Abstimmung gemäß \ref{grundsaetze:wahlen:wahlanfechtung},
  \item Wiederanerkennung eines Sitzes im Studierendenparlament  gemäß \ref{stupa:zusammensetzung:ausscheiden} Nummer \ref{stupa:zusammensetzung:ausscheiden:wiederanerkennung},
  \item Feststellung von Verstößen gegen die Organisationssatzung oder weiterer Satzungen,
  \item Prüfung der Fachschaftsordnungen.
  \end{enumerate}

Der Ältestenrat tagt mindestens einmal pro Vorlesungsmonat. Die Mitglieder sind zur Teilnahme an den Sitzungen verpflichtet.

Dem Studierendenparlament sind Protokolle der Sitzungen vorzulegen. Ein Mitglied des Ältestenrats soll ihm für Rückfragen zur Verfügung stehen.

Die Mitglieder des Ältestenrates haben in der Studierendenschaft uneingeschränktes Informationsrecht.

Eingaben an den Ältestenrat sind an die Vorsitzende zu richten. Sie versieht die Eingabe mit dem Eingangsdatum und veranlasst die Behandlung in der nächsten Sitzung. Über das Ergebnis ist die Eingebende zu unterrichten.

Ist der Ältestenrat mit zwei oder weniger Mitgliedern besetzt, so übernimmt das Präsidium des Studierendenparlaments im Einvernehmen mit den amtierenden Mitgliedern des Ältestenrats dessen Aufgaben nach \refPar{aera:aufgaben:allgemein} Nummer \ref{aera:aufgaben:vv} und \ref{aera:aufgaben:ua}.


\jurparagraph{Zusammensetzung}

Der Ältestenrat besteht aus fünf Mitgliedern. Sie werden vom Studierendenparlament auf ein Jahr gewählt. Die Amtszeiten der einzelnen Mitglieder beginnen entweder am 1. April oder 1. Oktober; sie sollen nicht alle am gleichen Datum beginnen.

Die Mitglieder des Ältestenrats sollen ehemalige Mitglieder der studentischen Selbstverwaltung sein.

Die Mitglieder des Ältestenrats dürfen weder Mitglieder eines anderen Organs der Studierendenschaft noch eines Organs des KIT sein oder für eines kandidieren.

Mitglieder des Ältestenrats scheiden aus
  \begin{enumerate}
  \item am Ende ihrer Amtszeit,
  \item durch Exmatrikulation,
  \item durch eigenen Verzicht,
  \item durch automatischen Ausschluss bei dreimaligem unentschuldigtem Fehlen bzw. bei insgesamt fünfmaliger Abwesenheit.
  \end{enumerate}
Bei vorzeitigem Ausscheiden eines Mitglieds erfolgt eine Nachwahl durch das Studierendenparlament für den Rest der Amtszeit.


\jurparagraph{Organisation}

Der Ältestenrat wählt sich seine Vorsitzende aus seiner Mitte.

Das Studierendenparlament kann auf Vorschlag des Ältestenrats eine Geschäftsordnung für den Ältestenrat beschließen. Ist eine solche nicht vorhanden, so findet die Ge\-schäfts\-ord\-nung des Studierendenparlaments sinngemäß Anwendung.


\jurparagraph{Beschlüsse}

Erklärt der Ältestenrat einen Beschluss eines Organs der Studierendenschaft für satzungswidrig, so ist dieser aufgehoben. Die Aufhebung eines Beschlusses ist schriftlich zu begründen und dem jeweiligen Organ mitzuteilen. Ein Mitglied des Ältestenrats soll dem jeweiligen Organ für Rückfragen zur Verfügung stehen.

Erklärt der Ältestenrat die Anfechtung einer Wahl oder Abstimmung für begründet, so veranlasst er die zur Behebung des Mangels erforderlichen Tätigkeiten. Kann der Mangel nicht behoben werden, so ist die Wahl oder Abstimmung ungültig und muss wiederholt werden.

Erhält der Ältestenrat den Antrag auf Wiederanerkennung eines Sitzes im Studierendenparlament, so gibt er der betroffenen Abgeordneten Gelegenheit zur Stellungnahme. Kann sie sich angemessen rechtfertigen, so erkennt der Ältestenrat den Sitz wieder an und teilt dies dem Präsidium des Studierendenparlaments mit.

%wann sind Beschlüsse des Ära gültig?

%
% FACHSCHAFTEN
%

\jursection{Fachschaften}

\jurparagraph{Aufgaben} \label{fachschaften:aufgaben}

Die Organe der Fachschaft nehmen die fakultätsbezogenen Studienangelegenheiten und Aufgaben im Sinne des \ref{studierendenschaft:aufgaben} auf Fakultätsebene wahr.

\jurparagraph{Gliederung, Mitgliedschaft}\label{fachschaften:gliederung}

Die Studierenden einer Fakultät bilden eine Fachschaft.

Die Fachschaften regeln ihre Angelegenheiten durch Fachschaftsordnungen selbst. Diese sollen dem Ältestenrat zur Prüfung vorgelegt werden. Fachschaftsordnungen sind vom Studierendenparlament als Satzungen zu beschließen.


\jurparagraph{Organe}

Organe der Fachschaft sind
  \begin{enumerate}
  \item der Fachschaftsvorstand,
  \item die Fachschaftsversammlung.
  \end{enumerate}

Die Fachschaftsordnung kann weitere Organe vorsehen.


\jurparagraph{Fachschaftsvorstand}\label{fs:vorstand}

Der Fachschaftsvorstand ist das ausführende Organ der Fachschaft. Näheres regelt die Fachschaftsordnung.

Der Fachschaftsvorstand besteht aus den Fach\-schafts\-sprech\-erinnen. Die Fach\-schafts\-sprech\-erinnen werden durch allgemeine, gleiche, geheime und direkte Wahl nach dem Grundsatz der Persönlichkeitswahl gewählt. Die Amtsperiode des Fachschaftsvorstandes beginnt in der Regel am 1. Oktober und endet am darauffolgenden 30. September. Es gelten die Vorschriften des \ref{grundsaetze:wahlen}. Näheres bestimmt die Wahl- und Abstimmungsordnung.

Die Anzahl der Fachschaftssprecherinnen wird unter Beachtung der Anzahl der Studierenden in der Fachschaftsordnung festgelegt. Sie beträgt mindestens zwei und höchstens acht. \label{fs:vorstand:anzahl}

%
% Beschluss FSK: bis 1000 Studis 2 FS-Sprecher, dann je angefangener 500 Studis einer mehr
%

Eine Fachschaftssprecherin scheidet aus dem Amt
  \begin{enumerate}
  \item am Ende der Amtsperiode,
  \item durch Exmatrikulation,
  \item durch eigenen Verzicht,
  \item bei Wahl eines neuen Vorstandes nach \ref{fachschaft:vv:wahl}.
\end{enumerate}

Bei Ausscheiden einer Fach\-schafts\-sprecherin rückt die Kandidatin mit den nächstmeisten Stimmen nach. Steht keine Kandidatin mehr zur Verfügung, bleibt das Amt unbesetzt. Fällt die Anzahl der Fach\-schafts\-sprecherin unter zwei, ist eine Fach\-schafts\-versammlung von der noch verbleibenden Fach\-schafts\-sprecherin innerhalb von zwei Wochen in der Vorlesungszeit einzuberufen, um über Neuwahlen zu entscheiden. Ist der Fach\-schafts\-vorstand unbesetzt, regelt die Fach\-schafts\-ordnung das weitere Vorgehen.

Die Fach\-schafts\-ordnung kann vorsehen, dass die jeweiligen studentischen Fakultätsratsmitglieder dem Fachschaftsvorstand angehören.

Die Mitglieder des Fachschaftsvorstands haben das Recht, Anfragen an den  Vorstand und das Studierendenparlament zu stellen. Anfragen sind  schriftlich an die Vorsitzende des betreffenden Organs zu richten.  Anfragen müssen vom Vorstand innerhalb von vier Wochen und vom Studierendenparlament innerhalb von vier Wochen während der Vorlesungszeit in Textform beantwortet werden.

Der Fachschaftsvorstand kann eine Person wählen, die mit beratender Stimme an den Sitzungen des Fakultätsrats teilnimmt.


\jurparagraph{Fachschaftsversammlung}\label{fachschaft:vv}

Die Fachschaftsversammlung ist das beschließende Organ der Fachschaft.

Jedes Fachschaftsmitglied ist auf der Fachschaftsversammlung stimm- und antragsberechtigt.

Die Fachschaftsversammlung wird mindestens einmal pro Semester und auf Antrag von mindestens 5~\% der Fach\-schaftsmitglieder vom Fachschaftsvorstand einberufen. Bei der Einberufung muss eine Tagesordnung vorgeschlagen sein. Die Fachschaftsordnung hat Regelungen zu Fristen und Bekanntmachungen zutreffen.

Die Fachschaftsversammlung kann Kompetenzen an andere Organe der Fachschaft übertragen. Folgende Kompetenzen sind nicht übertragbar \label{fachschaft:vv:kompetenzen}
  \begin{enumerate}
  \item Beschluss und Änderung der Fachschaftsordnung,
  \item Genehmigung des Haushaltsplans der Fachschaft,
  \item Beschluss einer Neuwahl des Fachschaftsvorstands gemäß \refPar{fachschaft:vv:wahl}, \label{fachschaft:vv:kompetenzen:abwahl}
  \item Einsetzen der Wahlleiterin.\label{fachschaft:vv:wahlleiter}
  \end{enumerate}

Die Fachschaftsversammlung kann mit 10~\% aller Stimmen und Zweidrittel der abgegebenen Stimmen be\-schlie\-ßen, eine Neuwahl des Fach\-schaftsvor\-stands zu veranlassen\label{fachschaft:vv:wahl}.

%
% FACHSCHAFTENKONFERENZ (FSK)
%

\jursection{Fachschaftenkonferenz}

\jurparagraph{Aufgaben} \label{fsk:aufgaben}

Die Fachschaftenkonferenz ist ein Organ der Studierendenschaft. Sie vertritt die Interessen der Fachschaften  gegenüber dem Studierendenparlament und dem Vorstand.

Die Fachschaftenkonferenz hat ein Vetorecht gegen Beschlüsse des Studierendenparlaments. Das Veto muss binnen einer Frist von zwei Wochen nach dem Beschluss im Studierendenparlament mit mehr als der Hälfte der satzungsgemäß existierenden Stimmen eingelegt werden. Durch Einlegen des Vetos wird der Beschluss des Studierendenparlaments aufgeschoben. Das Studierendenparlament kann ein Veto mit einer Zweidrittelmehrheit der Stimmberechtigten aufheben.\label{fsk:aufgaben:einspruch}

Legt die Fachschaftenkonferenz ein Veto gegen den Beschluss des Haushalts- oder Wirtschaftsplans ein, so muss sie zugleich einen alternativen Haushalts- oder Wirtschaftsplan beschließen. Über diesen alternativen Haushalts- oder Wirtschaftsplan ist vom Studierendenparlament innerhalb von zwei Wochen zu beschließen. Das Studierendenparlament kann einen neuen Haushalts- oder Wirtschaftsplan beschließen oder mit einer Zweidrittelmehrheit der Stimmberechtigten das Veto gegen den ursprünglichen Haushalts- oder Wirtschaftsplan aufheben. \label{fsk:aufgaben:haushalt}

Abweichend von \refPar{fsk:aufgaben:einspruch} kann das Studierendenparlament ein Veto nicht aufheben, sofern der Beschluss eine Änderung der §§ \refParagraphN{fachschaften:aufgaben} bis \refParagraphN{fsk:organisation} sowie \ref{haushalt:fachschaftsgelder} dieser Satzung beinhaltet.

Die Fachschaftenkonferenz wählt Vertreterinnen in den Finanzausschuss nach \ref{haushalt:finanzausschuss:wahl}.

\jurparagraph{Zusammensetzung, Stimmverteilung}

Die Fachschaften entsenden Vertreterinnen in die Fachschaftenkonferenz. Die Vertreterinnen jeder Fachschaft werden vom Fachschaftsvorstand gewählt und müssen von der Fachschaftsversammlung bestätigt werden.\label{fsk:zusammensetzung:vertreter}

Die Innenreferentin soll an den Sitzungen mit beratender Stimme teilnehmen.

Die Verteilung der Stimmen erfolgt unter Beachtung der Anzahl der Studierenden. Die Fachschaften mit
\begin{itemize}
\item bis zu 400 Studierenden haben zwei Stimmen,
\item von 401 bis 800 Studierenden haben drei Stimmen,
\item von 801 bis 1000 Studierenden haben vier Stimmen,
\item von 1001 bis 1300 Studierenden haben fünf Stimmen,
\item von 1301 bis 1600 Studierenden haben sechs Stimmen,
\item von 1601 bis 2000 Studierenden haben sieben Stimmen,
\item von 2001 bis 2500 Studierenden haben acht Stimmen,
\item über 2500 Studierenden haben neun Stimmen.
\end{itemize}

\jurparagraph{Organisation}\label{fsk:organisation}

Die Fachschaftenkonferenz gibt sich eine Geschäftsordnung.

Die Fachschaftenkonferenz wählt aus ihrer Mitte eine Präsidentin. Die Präsidentin ist für die ordnungsgemäße Einberufung und Durchführung der Sitzungen verantwortlich.

Antragsberechtigt sind
  \begin{enumerate}
  \item die Vertreterinnen der Fachschaften gemäß \ref{fsk:zusammensetzung:vertreter},
  \item der Vorstand der Studierendenschaft,
  \item die Fachschaftsvorstände,
  \item die Mitglieder nach Maßgabe von \ref{studierendenschaft:mitglieder:antraege}.
  \end{enumerate}

Die Fachschaftenkonferenz tagt mindestens einmal pro Vorlesungsmonat.

%
% ARBEITSKREISE UND HOCHSCHULGRUPPEN
%

\jursection{Arbeitskreise und Hochschulgruppen}

\jurparagraph{Arbeitskreise}
Zur langfristigen Bearbeitung konkreter Aufgaben oder Teile der Aufgaben nach \ref{studierendenschaft:aufgaben} kann das Studierendenparlament Arbeitskreise der Studierendenschaft einrichten. Diese sind dem Studierendenparlament weisungsgebunden und berichten diesem regelmäßig über ihre Arbeit.

\jurparagraph{Hochschulgruppen}
Studentische Gruppen haben die Möglichkeit, sich als Hochschulgruppe der Studierendenschaft beim Vorstand registrieren zu lassen. Voraussetzung sind eine Vereinbarkeit des Zwecks der Hochschulgruppe mit den Aufgaben der Studierendenschaft, dass der Schwerpunkt der Arbeit der Gruppe am KIT liegt und dass die Gruppe selbstlos tätig ist und nicht in erster Linie eigenwirtschaftliche Zwecke verfolgt. Näheres regelt eine gesonderte Satzung.

%
% HAUSHALT
%

\jursection{Haushalt}

\jurparagraph{Allgemeines}

Das Studierendenparlament hat die Verfügungsgewalt über das Vermögen der Studierendenschaft.

Das Haushaltsjahr der Studierendenschaft ist das Kalenderjahr.

Das Studierendenparlament erlässt eine Finanzordnung und eine Beitragsordung als Satzungen.

Die Fachschaften haben ein Anrecht auf 20~\% der Einnahmen durch Beiträge der Studierendenschaft. \label{haushalt:fachschaftsgelder}

Der Vorstand legt zum Ende des Geschäftsjahres dem Studierendenparlament und der Fachschaftenkonferenz eine Bilanz vor.

Der Haushalts- oder Wirtschaftsplan und die Bilanz werden veröffentlicht.


\jurparagraph{Haushalts- oder Wirtschaftsplan} \label{haushalt:haushaltsplan}

Der Vorstand legt dem Studierendenparlament spätestens bis zum 1.~Dezember einen Entwurf des Haushalts- oder Wirtschaftsplans für das folgende Geschäftsjahr vor.

Der Haushalts- oder Wirtschaftsplan muss für jedes Haushaltsjahr ausgeglichen sein.

Außer- und überplanmäßige Ausgaben müssen durch einen Nachtragshaushalt beschlossen werden.

Über das Eröffnen und Schließen von Geschäftsfeldern, sowie grundsätzliche Veränderungen der Wirtschaftsbetriebe, entscheidet das Studierendenparlament. Die Gründung von und die Beteiligung an wirtschaftlichen Unternehmen bedarf darüber hinaus der Zustimmung des Präsidiums des KIT.


\jurparagraph{Finanzausschuss} \label{haushalt:finanzauschuss}

Der Finanzausschuss unterstützt die Rechnungsprüfung nach §~65~b Absatz~3 Satz~2 LHG. Zusätzlich führt der Finanzausschuss eigene Prüfungen durch. Es erfolgt mindestens eine Prüfung im Semester; über das Ergebnis der Prüfung ist dem Studierendenparlament und der Fachschaftenkonferenz zu berichten. Näheres regelt die Finanzordnung.

Der  Finanzausschuss besteht aus drei durch das Studierendenparlament und  zwei durch die Fachschaftenkonferenz bestimmte Mitglieder. Sie werden nach Maßgabe der Finanzordnung auf ein Jahr gewählt. Die Mitglieder des Finanzausschusses dürfen nicht Mitglied des Vorstands oder erweiterten Vorstands sein. \label{haushalt:finanzausschuss:wahl}


%
% GRUNDSÄTZE
%

\jursection{Grundsätze und Organisatorisches}

\jurparagraph{Wahlen und Abstimmungen}\label{grundsaetze:wahlen}

Wahlen und Abstimmungen der Studierendenschaft finden nach demokratischen Grundsätzen statt. Die Einhaltung demokratischer Regeln ist durch eine geeignete Organisationsweise zu gewährleisten.

Verantwortlich für die Einhaltung demokratischer Regeln bei der Wahl zum Studierendenparlament und zu den Fachschaftsvorständen ist ein vom Studierendenparlament gewählter Wahlausschuss. Er wird bei der Durchführung von den Wahlleiterinnen der Fachschaften nach \ref{fachschaft:vv:kompetenzen} Nummer \ref{fachschaft:vv:wahlleiter} unterstützt. Unmittelbar nach Abschluss der Wahl oder Abstimmung ermittelt der zuständige Ausschuss das Ergebnis und hält es in einer Niederschrift fest, die dem Studierendenparlament und dem Ältestenrat vorgelegt werden muss. Außerdem sorgt er für die unverzügliche Bekanntmachung des Ergebnisses. \label{grundsaetze:wahlen:wahlausschuss}

Bekanntmachungen von Wahlen und Urabstimmungen sind vom Wahlausschuss öffentlich innerhalb des KIT auszuhängen. Mindestens ein Aushang an zentraler Stelle jeder Fakultät sowie der Mensa ist erforderlich. \label{grundsaetze:wahlen:bekanntmachung}

Jedes Mitglied kann eine Wahl oder Abstimmung beim Ältestenrat innerhalb einer Frist von vier Wochen ab der Bekanntmachung des Ergebnisses schriftlich anfechten. Erklärt der Ältestenrat die Wahl oder Abstimmung für ungültig, so ist die Wiederholung unverzüglich auszuschreiben. \label{grundsaetze:wahlen:wahlanfechtung}

Wahlen und Urabstimmungen finden während der vom KIT-Senat beschlossenen Vorlesungszeit an direkt aufeinander folgenden Werktagen statt.

\jurparagraph{Mehrheiten}
In der Regel ist ein Antrag angenommen, wenn ihm mehr anwesende Stimmberechtigte zustimmen, als ihn ablehnen (relative Mehrheit). Folgende Abweichungen von dieser Regel können in Satzungen oder Geschäftsordnungen vorgesehen sein:
\begin{enumerate}
\item Absolute Mehrheit, d.\,h. mehr Ja-Stimmen als die Hälfte der Anzahl der Stimmberechtigten,
\item Zweidrittelmehrheit der abgegebenen Stimmen, d.\,h. mindestens so viele Ja-Stimmen wie zwei Drittel der abgegebenen Stimmen,
\item Zweidrittelmehrheit der Stimmberechtigten, d.\,h. mindestens so viele Ja-Stimmen wie zwei Drittel der Anzahl der Stimmberechtigten.
\end{enumerate}
Als Anzahl der abgegebenen Stimmen gilt die Summe aus Ja-Stimmen, Nein-Stimmen, Enthaltungen und ungültigen Stimmen.


\jurparagraph{In-Kraft-Treten}\label{grundsaetze:inkrafttreten}
Diese Satzung tritt am Tage nach ihrer Bekanntmachung in den Amtlichen Bekanntmachungen des KIT in Kraft.


\end{jurdoc}
